\documentclass[11pt]{article}
\usepackage{textcomp}
\usepackage{fontenc}
\setlength{\parskip}{3 mm}

\renewcommand*{\familydefault}{\sfdefault}


\begin{document}

\title{Photoperiod and temperature control of spring phenology across woody plants}
\maketitle
\section*{Abstract}

\section*{Introduction}

A warming world is altering the 
Chamber experiment vs. observational study
Observational studies keep in situ ecological relationships intact, e.g. resource competition, and natural variation
Growth chambers provide fine-grained control of photoperiod, temperature, light intensity, CO2, humidity, and their respective interactions
There is not yet a quantitative review of photoperiod and temperature effects on spring phenology
We set out to aggregate and analyze published growth chamber experiment data to produce such a review


Basler \& Koerner 2012

Photoperiod sensitivity has been proposed to : Late successional app delayed by short photoperiod (= winter day lengths) while early successional no distinct pattern. 
Picea: elevation of origin mattered. 
Quercus and Abies: regional differences mattered

Basler \& Koerner 2014

endodormancy: internal
ecodormancy: inactivity imposed by env conditions
paradormancy: bud dormancy maininated due to physiological factors outside fo dormant meristems (apical dominance).

Factors of interest on leaf out 
Winter chilling
photoperiod
Warming
frost tolerance 

Test this proposition: If photoperiod is an important controller of spring phenology, we will observe an effect of photoperiod in aggregate across studies
Apply quantitative review to reconcile experimental studies with contrasting results of photoperiod and temperature in different species and environmental contexts

\section*{Methods}

We searched the literature for research papers which experimentally addressed controls of temperature, photoperiod, and/or chilling requirements on the spring phenology of woody plant species. We searched both ISI Web of Science and Google Scholar using the following search terms:
\begin{itemize}
\item{ (budburst OR leaf-out) AND (photoperiod or daylength) AND temperature* }
\item{ (budburst OR leaf-out) AND dorman*  }
\end{itemize}

For results of both searches, we then assessed if candidate studies met our criteria of focusing on woody plants in temperate ecosystems, and testing for at least photoperiod or temperature effects on budburst, leaf-out or flowering. Studies which included a chilling treatment 
 controlled environments, e.g. growth chambers
 
 201 papers in initial search, then narrowed down to 107 which fit our requirements.
 
 While most all studies measure time to burst, each may communicate results differently, e.g. days to budburst, degree-days to budburst, percent burst, number of leaves etc.
We account for both results and their respective units to ensure consistency in final R coding data analysis



\section*{Results}

\section*{Discussion}

\section*{Acknowledgements}

\section*{References}

\section*{Tables and Figures}

Figure 1: Effect sizes of temperature, photoperiod, chilling (and interactions) on advancement of phenology


\end{document}
\documentclass[10.5pt,a4paper]{letter}
\usepackage[top=.75in, bottom=.75in, left=.7in, right=0.7in]{geometry}
\usepackage{graphicx}
\usepackage{natbib}
\usepackage{gensymb}
\begin{footnotesize}
\address{1300 Centre Street \\ Boston, MA, 20131}
\end{footnotesize}
\begin{document}
\bibliographystyle{/Users/aileneettinger/citations/Bibtex/styles/nature.bst}
\begin{letter}{}
\includegraphics[width=0.3\textwidth]{/Users/aileneettinger/Dropbox/Documents/Work/AA_heading.pdf}
\pagenumbering{gobble}

\opening{Dear Dr. Sudgen:}
Please consider our paper, entitled ``Winter temperatures dominate spring phenological responses to warming'' for publication as a ``Report" in \emph{Science}. 

The timing of spring phenology (e.g., budburst, leafout) in woody plants is critical to plant fitness, shapes plant and animal communities, and affects wide-ranging ecosystem services from crop productivity to carbon sequestration. 
Advances in budburst are some of the most reported---and tangible---biological impacts of climate change, garnering great research and public interest. 

\par Recent warming has ignited debate over the fundamental drivers that determine spring phenology, with far-reaching implications for which environmental cues will dominate future trends  (\emph{1-4}).
%As interest has grown, so has debate over which environmental cues determine spring phenology, with far-reaching implications for forecasts and current trends  (\emph{1-4}). 
Although most temperate species show responses to spring warming (forcing), the prevalence and relative strengths of responses to chilling (associated with cool winter temperatures) and photoperiod (daylength) could slow or stall advances in spring phenology with continued warming. Indeed, recent work suggests chilling or daylength cues may underlie observed declines in the `temperature sensitivity' of leafout in Europe (\emph{5-6}). 
%\par The relative strength of these cues, which will determine how spring phenology shifts with future warming, is a source of controversy. Some research suggests photoperiod cues dominate in particular species or locations and are absent in others, whereas other research argues that winter temperatures (chilling) is critical and reductions in chilling caused by global warming are likely to have strong effects. Still other research highlights that all three cues may be important for the majority of species (\emph{5}). Reconciling these findings is challenging because studies employ divergent methodologies, across different species and locations, all of which may affect the strength of chilling, forcing, and photoperiod effects.

% We're too heavy on methods and light on resulst! So I tried to fix.
% Combining our meta-analytic approach with Bayesian hierarchical modelling allows use to extimate overall chilling, focing and photoperiod responses across 66 studies and 203 species, alongside species-specific estimates. .... To better understand how the model estimates relate to conditions in the real world, as opposed to experimental ones, we apply our model to well-studied locations in central Europe. Using simple forecasts of warming and historic climate and budburst data, we find that the impact of chilling and daylength cues is highly location-specific---dependent largely on whether chilling increases or decreases with warming. Thus, our results may serve to unify opposing sides of the controversy over phenological cues: while all species may respond to all cues in experimental conditions, the dominant impact of climate change appears to be from increased forcing in current environmental conditions. 
\par We address this controversy by synthesizing four decades of controlled environment experiments to estimate overall chilling, forcing and photoperiod responses across 66 studies and 203 species from around the globe. We find most species respond strongly to all three cues, with chilling being the strongest cue---nearly four times greater than forcing. Yet, when we applied our results to areas with reported declining phenological responses to warming (\emph{5}), we find few sites where chilling or daylength cues would constrain leafout advances under current or near-term warming. Instead, we suggest observed declines may be due in part to a statistical artifact: we show that temperature sensitivities (measured in days per $^{\circ}$C) calculated without correcting for warmer daily temperatures will always predict a decline with warming---even with no change in chilling, forcing or photoperiod responses. % an artifact with wide-reaching implications for many areas of climate change plant biology. 
Our results thus resolve several major debates in plant phenology by showing that most species respond to all cues strongly in experimental conditions, but forcing appears to determine responses to recent warming. % Thus, spring may continue to advance in many well-studied European regions in the future with the most dramatic changes coming from regions were winter warming causes dramatic decreases in chilling, with implications for ecosystem services related to phenology. 

\par Upon acceptance for publication, the database will be freely available at KNB (\emph{7}; currently meta-data are there); the full database is available to reviewers and editors upon request. This work is a meta-analysis, so data have been previously published; however, the synthesis of these data and the tables, figures, models, and materials presented in this manuscript have not been previously published nor are they under consideration for publication elsewhere.

T. J. Davies, S. Elmendorf and  J. Hille Ris Lambers have previously reviewed the manuscript. We recommend the following reviewers: Josep Pe\~nuelas, David Inouye, Ally Phillimore, and Mark Schwartz. \\

Sincerely,\\

\includegraphics[scale=.4]{/Users/aileneettinger/Dropbox/Documents/Work/AileneEttingerSignature.png} \\
Ailene Ettinger\\
\begin{footnotesize}
Visiting Researcher, Arnold Arboretum of Harvard University 

\newpage
\noindent \emph{References mentioned in cover letter}

\begin{enumerate}
\item K\"orner, C., \& Basler, D. 2010. Warming, photoperiods, and tree phenology response. \emph{Science}, 329: 278-278.
\item Chuine, I., Morin, X., \& Bugmann, H. 2010. Warming, photoperiods, and tree phenology. \emph{Science}, 329: 277-278.
\item Zohner, C. M., et al. 2016. Day length unlikely to constrain climate-driven shifts in leaf-out times of northern woody plants. \emph{Nature Climate Change}, 6: 1120.
\item Flynn, D. F. B., \& Wolkovich, E. M. 2018. Temperature and photoperiod drive spring phenology across all species in a temperate forest community. \emph{New Phytologist}, 219: 1353-1362.
\item Fu, Y. H., et al. 2015. Declining global warming effects on the phenology of spring leaf unfolding." Nature 526: 104.
\item Richardson, A.D., et al. 2018. Ecosystem warming extends vegetation activity but heightens vulnerability to cold temperatures.  \emph{Nature}, 560: 368.
\item Wolkovich, E., et al. 2019. Observed Spring Phenology Responses in Experimental Environments (OSPREE). Knowledge Network for Biocomplexity. urn:uuid:b2ab2746-b830-436b-a7a9-01b3ef3558e4. 
\end{enumerate}
\end{footnotesize}



\end{letter}
\end{document}

\item Yu, H., Luedeling, E., \& Xu, J. (2010). Winter and spring warming result in delayed spring phenology on the Tibetan Plateau. \emph{Proceedings of the National Academy of Sciences}, 107(51), 22151-22156.

  %  A cover letter which should include
 %       Reference to any pre-submission discussions with editors.
 %       The title of the paper and a statement of its main point.
   %     Any information needed to ensure a fair review process, including related manuscripts submitted to other journals.
     %   Names of colleagues who have reviewed the paper.
        %Specification of where all data underlying the study are available, or will be deposited, and whether there are any restrictions on data availability such as an MTA.

 %       Please also upload a .docx version of your cover letter ? see below.
    %You will have the opportunity to request a specific editor, but this is not required and editor assignment also depends on availability, relative loads and other factors.
    %We require you to list all funding sources. This can be done through a dropdown if your funder is included in FundRef?s controlled vocabulary list.

%Reviewers: Names, affiliations, and e-mail addresses of up to five potential reviewers and up to five excluded reviewers.


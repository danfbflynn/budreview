\documentclass[11.5pt,a4paper]{letter}
\usepackage[top=.75in, bottom=.75in, left=1in, right=1in]{geometry}
\usepackage{graphicx}
\usepackage{natbib}
\usepackage{gensymb}
\begin{footnotesize}
\address{1300 Centre Street \\ Boston, MA, 20131}
\end{footnotesize}
\begin{document}
\bibliographystyle{/Users/aileneettinger/citations/Bibtex/styles/nature.bst}
\begin{letter}{}
\includegraphics[width=0.3\textwidth]{/Users/aileneettinger/Dropbox/Documents/Work/AA_heading.pdf}
\pagenumbering{gobble}

\opening{Dear Dr. Findlay:}
Please consider our paper, entitled ``Winter temperatures dominate spring phenological responses to warming'' for publication as a ``Letter" in \emph{Nature Climate Change}. This manuscript is a revised version of an earlier submission (NCLIM-19081773). We include a point-by-point response to reviewer comments. 

\par As you may recall, our manuscript utilizes an extensive new, extensive, global database to address a research topic of critical relevance to a broad swath of \emph{Nature Climate Change} readers:  the timing of spring phenology (e.g., budburst, leafout) in woody plants. Spring phenology impacts plant fitness, shapes plant and animal communities, and affects wide-ranging ecosystem services from crop productivity to carbon sequestration. Our work is groundbreaking in its synthesis of four decades of research across 72 experiments
to quantify the relative importance of three environmental cues critical to phenology. We estimate overall chilling, forcing and photoperiod responses for 203 species from around the globe. 

\par The three reviewers recognized the potential of our work to influence future research, as well as its interest to \emph{Nature Climate Change} readers. They also highlighted some concerns. Reviewer 1 suggested that additional details and clarification of methods would be beneficial for a fuller evaluation of the study.  Reviewer 2 felt unconvinced that the experimental methods synthesized in our meta-analysis could be reliably applied to natural systems. Reviewer 3 had reservations about the validity of the results given the data and modeling approaches used. 

% EMW: eventually here we want to give some sense in the topic sentence of how much as changed -- changed x% of the text, did Y new analyses, added this many new main text and supp figures. And then also state somewhere that this work has shown our results to be robust (i.e., somewhere you need to say that our big important point still stands). 
\par We have substantially modified the manuscript to address the concerns expressed by reviewers and the issues mentioned by the Editor after the initial submission. Specifically, we have added new text to the main manuscript, including methodological details, XX, and XX. We have also created a new figure, and modified previous figures in the main text to addres reviewer concerns. We have also added the online `Methods' section to adhere to the new guidelines of \emph{Nature Climate Change}.

\par Upon acceptance for publication, the database will be freely available at KNB (\emph{7}; currently meta-data are there); the full database is available to reviewers and editors upon request. This work is a meta-analysis, so data have been previously published; however, the synthesis of these data and the tables, figures, models, and materials presented in this manuscript have not been previously published nor are they under consideration for publication elsewhere.


Sincerely,\\

\includegraphics[scale=.4]{/Users/aileneettinger/Dropbox/Documents/Work/AileneEttingerSignature.png} \\
Ailene Ettinger\\
\begin{footnotesize}
Quantitative Ecologist, The Nature Conservancy- Washington Field Office
Visiting Fellow, Arnold Arboretum of Harvard University 

 \end{footnotesize}
\clearpage

\title{Response to Reviewers}
 \emph{Reviewer Comments are in italics.} Author responses are in plain text.

 
 \emph{\bold{Reviewer #1 (Remarks to the Author)}}

\emph{The relative importance of forcing, chilling and photoperiod as cues for budburst is a fascinating one, with clear implications for predicting how species will respond to climate change. Here the authors leverage an exceptional dataset arising from experimental studies using sophisticated statistical analyses and arrive at the surprising conclusion that plants are generally more sensitive to chilling than forcing. I think this study has the potential to make a really valuable contribution that will be of broad interest to readers of this journal. However, I have quite a lot of criticisms/concerns of the study as it stands.}

\par We thank the reviewer for the recognition that the OSPREE dataset is exceptional and that our study can make a valuable contribution to \emph{Nature Climate Change} readers. We have revised the original manuscript substantially to address the reviewer's concerns, as detailed below. 

\emph{(1) Models: The STAN modelling approach is sophisticated but I think the model is rather incomplete and this could affect the inferences that are reached. For instance why aren't terms included to allow the intercepts and slopes to vary across studies within species?}

\par We completely agree with the reviewer that, ideally, our models would account for both variation in budburst responses among species and among studies. This was our aim in building the OSPREE database and using Bayesian models that can allow complex models to converge. During our data and model development, however, we found studies were generally confounded with species. This is a common problem in this area of research and other meta-analyses have faced similar issues (e.g., Kharouba et al 2019). Without strong priors to help differentiate what variation should load onto the study versus species, it is difficult to fit both variables. Indeed, we evaluated models that included studies within species and studies crossed with species but both models were found to be unstable (e.g., estimates varied widely across model runs, model estimates had wide credible intervals, chains frequently did not converge). \\

\par To address this we worked to present a main model that only included species for which we had multiple treatments across multiple studies. We combined species found in only one study into ``complexes" at the level of genera, such that each taxonomic unit we use in our model occurs across multiple studies (and treatments). Thus our taxonomic units of analysis
are ``species complexes," which are either species represented in >1 dataset or complexes combining multiple species within a genus that are each singly represented in the dataset. Species represented in only one dataset with no con-generics in other datasets were excluded from most of our analyses. We do include a model fit to all species and report these results in the supplement (page XX); we refer to this model as the `all species' model.

\par We realize now we should have made these issues more clear. We now write on lines XX:
`Some species are often only represented in one dataset in the OSPREE database, making it impossible to statistically differentiate between species, study, and treatment effects for these taxa. To address this, we combined species found in only one study into ``complexes" at the level of genera---such that each taxonomic unit we use in our model occurs across multiple studies (and treatments, see the \emph{The Observed Spring Phenology Responses in Experimental Environments (OSPREE) database} section in the Supplemental Materials for details.'

\emph{Also, I would have thought there is very likely a geographic effect on the effects, and I suggest that you test whether the results are sensitive to inclusion of a spatial random term across which slopes and intercepts vary.}


\par The reviewer makes an excellent point that budburst responses to temperature and photoperiod may vary due to the spatial location of studies or the geographic origin of plant material. In particular,  budburst responses are expected to vary by latitude (11; 12; 13). We address variation in space by including a latitude model, which we now discuss on lines XX in the main manuscript.  We focus on the spatial effect of latitude because there is strong evidence that there can be latitudinal differences in budburst responses, for example via interactions between latitude and photoperiod sensitivity (XX CITATIONS) and AND interactions with CHILLING?FORCING?) responses. we also provide the OSPREE database for researchers wishing to test other geographic effects.'

% Yes! The more analyses we can add to bolster our results, the stronger our reply is. Remember we only have chilling for NA and Europe so I think our options are this as a fixed effect or a model of bb ~ photo + force with continent on the intercept (latter seems better to me, include just in supp). 
\par QUESTION: Do we want to add a model with continent (or UN units: Africa, Americas, Asia and the Pacific, Europe and Central Asia, and the Middle East) as a random effect? we could at least interpret the variation at this level.

\emph{(2) Meta-analysis: The analysis is described as a meta-analysis, but falls short of being a formal meta-analysis as it seems as though measurement error in the response variable is not incorporated. This should be straightforward to incorporate and I was surprised that it hadn't been given the complexity of the analyses. Also, please report the extent to which the approach followed recommendations made in the PRISMA checklist }

\par This is a good point; we now review the PRISMA checklist point-by-point in the Supp (lines xx-xx). Many of the checklist items are done by our data publication, but we feel this addition greatly strengthens the paper and appreciate the reviewer suggesting it.

\emph{(3) Methods: The methods seem to be missing from the main ms, and I kept flicking forward to consult a section that does not exist. I thought the Nature letter format does allow a methods section and I found it really to the detriment of the readability of the ms that there wasn't one.}

\par We appreciate the reviewer's concern that methods are not easy to find; this was a concern of multiple reviewers so clearly something we needed to improve. We have worked to now more clearly embed key methods in the text (changes on linex xx-xx, yy-yy etc.).  ... 

\par We also provide a separate Methods section with full details of the data and are analyses. This section will be available online, should our manuscript be accepted for publication in \emph{Nature climate change}, following the journal's requirements.

\emph{(4) Chilling, forcing and photoperiod: In order for a reader to reach a conclusion about the robustness of the inferences it is vital that the method for quantifying cues is easily understandable. Currently in the main ms it is not (last paragraph of page 3). For instance, we are told the minimum temperature for chilling but not maximum, we are not informed as to when the chilling and forcing periods are and no discussion is given as to how the effect of photoperiod is modeled. It's also unclear in the main ms what a `standard unit' (I see it is described in the supplement) is and this leaves the reader disconnected with what the analyses are doing. A simple remedy for this would be to include a schematic (as figure 1) that identifies the information used to quantify each cue and relate it to the response. In general the main ms does a very poor job of explaining what was done (the data used, how cues were inferred and vital details about what the models were estimating), instead referring the reader repeatedly to supplementary materials. While the supplementary materials are generally good I still felt disconnected from the data and how the cues were actually quantified. This could be addressed by taking some example datasets and working through in detail how the different metrics were calculated. Without knowing what was done I find it very hard to judge whether the main conclusions are robust.}


\par We thank the reviewer for pointing out that important details on the chilling, forcing, and photoperiod estimates used in our analysis were unclear in the original version. We have added a new schematic figure (the new Figure 1), as the reviewer suggested, which we hope clarifies how the chilling, forcing, and photoperiod estimates were obtained from the original studies for use in our meta-analytical work. We have added relevant details on chilling, forcing, and photoperiod to the main text in the following locations:

\par We also provide full descriptions, including the upper and lower thresholds for chilling, in the online methods section (NEED TO ADD THIS).

\emph{(5) Chilling: I think it's important to know whether the inferences are robust to an alternative model of chilling, e.g., the sequential model that is widely used. From the supplementary materials it is clear that some effort has been made to consider alternatives (chilling portions) but given this analysis underlies the main conclusion of the paper I'd like to see alternative hypotheses considered.}

% EMW: If we want to do even more we could add chill days (days between 0-10 C) but I think the above could be enough (if we are addressing most reviewers concerns---it always comes down to a balance, you can argue some concerns as long as you have really addressed enough of the other ones). 
\par This is an excellent point as our earlier version did not adequately compare results with the Utah model to results with the other chilling model we evaluated, Chill Portions. Now we more clearly compare them in the main text (lines xx-xx). [GIVE QUOTE of main text \begin{quote} and \end{quote}]. Based on our understanding and in consultation with Isabelle Chuine the sequential model usually refers only to process-based models thus we are unsure how to apply it in our framework; if the reviewer has specific modeling suggestions or other chilling model he would like applied we would be happy to test them.

Possibly: We have also added a supplemental figure showing chill portions (Figure SX) - compare this to Figures 2-3 (should we add this? not sure its necessary...) 
Ailene needs to look at this more. we do present both....perhaps 3d figures for both in supp?

\emph{(6) Estimates: It is surprising to see point estimates repeatedly reported throughout the ms without 95\% credible intervals, this needs to be rectified. Also at present there is no formal test of whether the chilling response is significantly stronger than the forcing response, though this would be easy to do using the posteriors.}

\par We thank the reviewer for this suggestion and  have added 96\% credible intervals to all estimates presented in the main text (e.g., Lines XX-, Lines XX) .  In figures, we show 50\% credible intervals, and in the supplemental tables we present both 50\% and 96\% credible intervals.  We use 96\% credible intervals rather than the more frequently used 95\% credible intervals because....Our approach is bayesian and cite some of Gelman (or some one else's) stuff that 95\% = arbitrary

\emph{Statistical artefact with linear regression (Page 5): That application of linear regression to data arising from a growing degree model can lead to biased estimates is a fascinating insight. However, in the supplementary materials it is not clear to me how the temperature sensitivity window for linear regression (for B. pendula or the simulations) is calculated/defined. 
\par The window we used was March 1- June 1. 

How much can the issue of an advancing period of sensitivity be addressed by allowing the sliding window to shift over time? This issue is discussed in Simmonds, E. G., Cole, E. F., \& Sheldon, B. C. (2019). Cue identification in phenology: a case study of the predictive performance of current statistical tools. Journal of Animal Ecology.}

-cite this paper somewhere in our manuscript. describe the window used (how many days before...I think it started sept 1)
-sept 1. Dan and Cat are looking at effect of sliding window?
- we agree this is an important issue. more research needed.

Minor comments

\emph{Page 2. I suggest changing 'high unexplained variation across' to 'substantial variation among'.}

\par  We thank the reviewer for this suggestion and have made the recommended change (now Line XX).
 
\emph{Page 3. All three cues are not generally correlated in longitudinal studies; photoperiod and forcing are, but neither is usually very correlated with chilling.}

We thank the reviewer for pointing out that our writing was not clear in this section. In our earlier version of the manuscript, we mentioned correlations between cues but did not specify clearly whether we meant correlations across space or time, nor were we clear about the scale or window at which these cues can be correlated. Though the reviewer states that chilling is often not correlated longitudinally, we have found that it can be correlated (Cite figure in supplement?). In addition, chilling and forcing are frequently negatively correlated in space (sites that have experience high chilling tend to have low forcing). In this new version of the manuscript, we have clarified our writing, which now says (Lines XX):
`` XX''

\emph{Page 3. Last sentence of paragraph 2. This is hyperbole. The mean is not expected to shift far beyond historical bounds, though the extremes clearly will.}

The reviewer appears to have concerns about the following phrase from the previous version of our manuscript: ``... continued warming pushes climate into environmental regimes far beyond historical bounds.'' We thank the reviewer for highlighting this phrase, which is a bit vague in its reference to `environmental regimes' and for which we clearly should have included citations to support. We now cite the IPCC forecasts that climate change is expected to push temperatures to XXX, well beyond the historical bounds. We adjusted the sentence to be more clear and have added references so that it now says (Lines XX): ``Resolving these discrepancies is critical to accurate predictions of spring phenology, especially as continued climate change will yield warmer temperatures than has been experienced in at least 150 years and warming will occur at a rapid pace (Ohlemuller et al 2006,Williams et al 2007, Williams et al 2007b,IPCC 2013, Xu et al 2018).

\emph{Page 3. Fourth line from bottom. Is interactions the correct term?}

We apologize that this was unclear. We have now adjusted this sentence (lines xx-xx), which now reads ``Our model averages over interactive effects of predictors, as it includes only main effects; it estimates both species-level responses..." 

 \emph{Reviewer #2 (Remarks to the Author):}

\emph{Spring leaf-out phenology plays a key role in terrestrial carbon and water flux, but the underlying processes are still unclear, especially how the environmental cues, including chilling, photoperiod, and spring warm temperatures, interact and determine the leaf-out processes is still unclear, although most of the phenologist agreed that these three cues are all important. Therefore, quantify the relative importance are valuable and might be important for the phenology modeling and dynamics vegetation models. I carefully read this meta-analysis and found this is an interesting study, but I'm wondering, given the results were reliable, whether the meta-analysis results across experimental studies could reflect the natural plants' response? Or could we rely these experimental results that may inaccurate reflect underlying mechanisms? Because, according to the author (E.M. Wolkovich) previous study, the phenology under warming experiments could not reflect the natural observations (Wolkovich et al, 2012 nature, warming experiments underpredict plant phenological responses to climate change), which might arise from complex interactions among multiple drivers and remediable artefacts in the experiments that result in lower irradiance and drier soils.}

\par We thank the reviewer for pointing out that there are limits to the controlled environment studies synthesized in our meta-analysis. We agree that there are many ways in which experimental conditions differ from observational conditions and we were not clear about this in our previous version. The reviewer's comment also highlights that the previous version did not adequately describe the experimental methods used in controlled environment studies. 
\par To address these comments, we have added a new figure (Figure 1) to show the experimental design of the controlled environment studies in our synthesis. We have also added the following text (Lines XX):
``In contrast to field warming experiments, which are designed to test higher temperatures in natural conditions, these experiments are designed to create conditions less often seen in nature, but which can critically help researchers identify cues."
\par  We hope that these changes clarify that the controlled environment studies in the present paper differ from  the field-based warming experiments in Wolkovich et al 2012. Controlled environment studies are typically designed to tease apart the role of different cues (chilling, photoperiod, forcing), rather than to replicate current or future natural conditions (the warming studies synthesized in Wolkovich et al 2012  are typically designed to replicate natural conditions).

%we highlight this with our chilling comparison- experiments use constant chilling vs natural conditions were temp. 
%perhaps a references that responses are correlated in experiments and natural world? (is there a primack or zonner paper that does this?)
%foundational work on chiliing- - the whole idea of this process is an established thing. this are established methods developed because they work.

\emph{Furthermore, I'm not convinced that the chilling overweight forcing, and the effect of chilling, photoperiod and forcing might be quantified across more than 200 species based on the various manipulative experiments and MCMC-based Bayesian method, especially considering most of these experimental studies conducted only one year or less than 3 years. The main reasons come from: 1) most of these experimental studies conducted with very different settings, such as using saplings vs. mature tree?s cuttings, how the ontogenetic effects play a role or impacts the results? Arbitrary controls in lights/photoperiod length/intensity vs. greenhouse natural light; in addition, for many experimental studies, the temperature and photoperiod were set under extreme climates. I would say this is a response to extreme climate. All of these factors might substantially affect the results.}\\

\par We agree with the reviewer that there are many factors, not included in our main model, that could explain variation in budburst responses to chilling, photoperiod, and forcing (e.g., ontogeny). In this version we have tried to more strongly indicate this, as well as state that separating chilling from forcing requires more physiological research. For example, please see Lines XXX, where we state:
``''
\par In addition, in this new version we include a new model to directly test for ontogenetic effects on budburst by adding a predictor of "life stage" (juvenile vs. adult) to the main budburst model. We found that material from juvenile trees (seedlings or saplings) burst, on average, 3.6 days later than material from adult trees. We have added this model and a table summarizing its results to the supplemental materials. 
\par We believe that there are many additional potential questions and avenues of research to better understand how woody plant phenology responds to different cues, at varying life stages, and in different contexts. Our database and analyses provide a critical step in pushing this work forward by providing the first comprehensive meta-analysis quantifying responses to chilling, photoperiod, and forcing; it is our hope that our analyses, as well as the freely available OSPREE database, stimulates future experiments and analyses to test additional hypotheses.

\emph{2) the interact between chilling, photoperiod and forcing is complicated, and there are still unclear in many important facts. For example, the temperature thresholds of chilling and forcing estimation, and its species-specific values, are largely unknown. For some boreal or alps plants, they may budburst even when air temperature around freezing points, but the temperate trees are still dormancy even air T > 15 degree; the correlations between eco- and endo-dormancy, corresponding the chilling and forcing, whether they are a parallel or a sequential pattern between chilling and forcing? When/how the photoperiod plays its role during the two phase dormancy? Once the endo-dormancy break, continuous chilling accumulation, for example a cold span during spring, is still active? Or entirely depending on the forcing? All these questions are still not figured out; 3) except chilling, forcing and photoperiod, other cues are also involved with the leaf-out processes, for example air humidity, see Laube et al, 2014 (but recently, Zohner et al, 2019 New phytologist deny this effect) and soil moisture and snow cover. Under manipulative conditions, these effect might be largely ignored as argued in Wolkovich et al, 2012 as well.}

\par We completely agree with the reviewer that the interaction between chilling, photoperiod and forcing is complicated, as we note on lines xx-xx, yy-yy and such. We have now tried to address some of the complexities in our new Figure 1 and lines xx-xx ... [then say] We note however, that our results are very strong, suggesting some of these complexities are outeighed by the current evidence coming from these studied [okay, we can say that better, but hopefully you get my jist.]

\emph{4) species-specific response to chilling, photoperiod and forcing. This has been well reported, for example the pioneer species are opportunistic and photoperiod-insensitive, in contrast the late successional species are sensitive to photoperiod and higher forcing requirements, see the papers, as the authors cited, K�rner \& Basler 2010;2014; Laube et al, 2014; Zohner et al, 2016 and other studies. Across so large dataset/many species, the mean values, for example chilling effect is 2 times larger than forcing and photoperiod as well as its sensitivity, hold large uncertainty and are no sense.}

\par We completely agree with the reviewer that species-specific differences are important. Indeed our modeling approach is designed to help examine across-species effects but also species-level differences. We now clarify this on lines xx-xx (insert quote maybe). Our modeling approach is amazing, it does this....

\emph{One of the main conclusions is that chilling is over-weight forcing and recent advanced leaf-out is mainly associated with spring warming. However, this is inconsistent with recent study that found the spring phenology did not significantly change during the global warming hiatus, see its figure 1 in Wang et al, 2019 Nature comm, but the spring T is still significantly increase and winter getting colder over the Eurasian (Li, Stevens and Marotzke 2015 GRL)). It seems that increasing chilling and forcing could not explain the dynamics in spring phenology? How to explain this inconsistency?}

% EMW: Response seems on the right track, I can try to help with wording more later. 
- It seems that the reviewer is saying.....IfNot sure exactly what the reivewier's point is- that winters got cooler so we would expect spring phenology to advance if our hypothesis is correct? This is not necessarily true because we find that wamring is increaing chilling in mayn locations. Thus, cooling might be decreasing chilling- still could be consistent with the Wang et al paper. this actually may be consistent with our findings.

% EMW: Seems good on the below minor comments -- just aim for them (mostly) to read as 'Good point, done (lines xx-xx).'
\emph{Minor commons}
\emph{Line numbers are needed;}
\par We thank the reviewer for this suggestion and have added line numbers. 

\emph{In methods, the study yielded data from 72 studies across 39 yrs... this is misleading, because for many experimental studies, table S1, the data only for one year, and most less than 3 yrs.} 
To address this concern, we have added the phrase `with most studies lasting 1 year' so that the phrase now says (Line 51):
`...yielding data from 72 studies across 39 years and 203 species (with most studies lasting one year, Fig....' 

\emph{More description is needed of Bayesian hierarchical model in the main text;}
% EMW: We can add an in-line equation! It's simple enough to fit I think. 
how to address  this when no methods in main ms?!?

\emph{In the results sections, chilling has greater effect on budburst than forcing?. I would suggest providing the conditions, i.e. under future climate warming, due to the fact that these results come from experimental studies that simulated future warming, }
-sure- can say this ore carefully
\emph{In the results sections as well, the chilling only occur at warming above 4�C? interesting, but does it occur across species? and locations?} 
% EMW: Yes! Adding figures to supp would be good here! We want to win this reviewer over to our paper. 
-mean across species- explore variation? see maps in supplement

\emph{Zohner, Constantin M., et al. "Rising air humidity during spring does not trigger leaf?out timing in temperate woody plants." New Phytologist (2019).
Wang, Xufeng, et al. "No trends in spring and autumn phenology during the global warming hiatus." Nature communications 10.1 (2019): 2389.
Li, Chao, Bjorn Stevens, and Jochem Marotzke. "Eurasian winter cooling in the warming hiatus of 1998?2012." Geophysical Research Letters 42.19 (2015): 8131-8139.}
-cite these in main ms
-could add humidity when to main text- 
 \emph{Reviewer #3 (Remarks to the Author):}

\emph{This manuscript addresses the relative importance of the environmental determinants of plant phenology using a meta-analytical approach. Specifically, the authors combine the experimental results of 72 studies and 203 species to estimate the effects of day length, winter chilling, and forcing on spring phenology, using hierarchical Bayesian models. The main finding is that almost all species respond to all three cues, with chilling having the largest, day length the smallest effect. Furthermore, the results suggest that, while all cues are important under experimental conditions, spring forcing will remain the dominant driver of spring phenology over the coming decades. The manuscript is well written and addresses a clear question. However, I have reservations as to the overall importance and validity of these results. That chilling is more important than day length has been shown by previous multi-species studies addressing this (e.g., Laube et al. 2014, Zohner et al. 2016).}
% EMW: Good reply below, I might start with a quick -- we thank the reviewer for his/her review and appreciate s/he found the manuscript well-written and addressing a clear question. 
-true! other studies have looked at this. this is a meta-analytic approach. also, non of these studies attemted to look at 3 cues as tthe reviewer points outBut forcing is usually considered to be the most important!

\emph{Furthermore, the model output seems to suggest that all three cues (day length, chilling, and forcing) affect phenology in almost all species, leading the authors to conclude that their results contrast with the extensive literature [Zohner et al. 2016, K\"{o}rner \& Basler 2010*] suggesting photoperiod is an unimportant cue for many species. [page 4]? Yet, when looking at Table S2, most of the species-level data they use are taken from Zohner et al. (2016) [Zohner16 database]. In fact, 173 (85\%) of the 203 species included in this study were already investigated in Zohner et al. (2016). Given that in Zohner et al. (2016), 112 (65\%) out of 173 studied species did not react to daylength at all, it is surprising that day length is reported as a relevant, consistent cue across species. This makes me wonder whether their hierarchical Bayesian model is confounded (e.g., giving to much weight to certain species ?complexes?) and thus not suitable for exploring the relative importance of the different environmental drivers of spring phenology.}

% EMW: Most of your points here seem good, I think the reviewer is definitely focusing on 144/203 species so we should ideally open with 'this is a good point and showed that we did not do a good job clarifying how we addressed the complexity of species and studies in our meta-analysis (an issue also raised by R1). We now do X in the text to address this.

% EMW: Then I would also show some tests that you add to the supp -- I think the most useful is a similar model MINUS zohner, but zohner alone could also work. I suspect some of the difference is treating time as a factor (Zohner) versus calculating field-chilling (our approach) and I think the reviewer might appreciate this comparison. So, what I would do is try to the model without Zohner and aim to include that in supp (and cite in main text with pointing out species issue the reviewer raises) and then also try a Zohner-only model. If it makes sense, great! Let's include it. But if it doesn't make sense I suggest we not start digging in -- it's not our job to re-do Zohner. 
-interesting point about species differences. this is why we focus on the but compare only really abundance species: betula, fagus, quercus and
- not sure what the reviewer means when s/he says that "most of the species-level data" are from Zohner. The Zohner dataset comprised 864/7459 (11\%) rows in OSPREE. 
ITs true that the zohner dataset includes 144 out of 203 species in the full OSPREE budburst database. However, we include in the main model interpreted and presented  in the figures only ispecies that wer across multiple studies. 
- To make this more clear, we have added columsn to S2 which lists models included in?
-many zohner species excluded frmo main bb model. in model that includes all species (Table S3)- estimated effect of photoperiod does weaken (cforcing estimate gets stronger, chilling is about the same)
-add comparison between all species and single model
-look at zohner methods- could analyze this alone? oucld be because we estimated. 
- add somehing about partial pooling.- the reviewer seems to not understand this. 

\emph{K\"{o}rner \& Basler 2010 clearly is an inadequate reference here, please delete}
% EMW -- are you sure? Both these articles are flimsy short reviews, so I would just say -- we understand the reviewer's concern and now cite papers with more data [in both cases! K&B and the perspective]. If you need citations, check the two papers.
\par In our original submission, we cited two papers by K\"{o}rner \& Basler 2010: a `Perspective' in \emph{Science} (K\"{o}rner \& Basler 2010,  \emph{Science} 327, 1461) and a response to a critique of this perspective (K\"{o}rner \& Basler 2010, \emph{Science} 329, 278). We believe that the reviewer is referring to the response and we agree that this reference is inadequte. We thank the reviewer for this comment, and have removed the reference from our manuscript. 

\emph{Apart from that, I take issue with the estimation of the importance of forcing and the attempt to estimate the relative importance of day length, chilling, and forcing. First, I don't see how the effect of forcing can be disentangled from the effects of chilling. This would require knowledge on which temperature ranges are adequate to satisfy chilling and forcing requirements. Yet, as correctly stated in the Supplementary information (page 2), current models of chilling are hypotheses and likely to be inaccurate for many species. Similarly, the effective temperature ranges to fulfill forcing requirements are not known. As such, when comparing the relative importance of winter chilling versus spring warming both factors are likely to be confounded. Also, if a study uses two different forcing temperatures that both lie within the range of optimal forcing conditions, one would see no effect between the treatments and the authors would thus infer that forcing didn?t affect phenology, when in fact, forcing has a huge effect, not detected by the study design. Given these considerations, I don't think that a multivariate model, such as the one presented in this study, can adequately disentangle the relative importance of the three main phenological cues. }
 
% EMW: Seems good -- main reply should be: we agree, we address this with (1) new figure, (2)  moved the supp point (and ref) about this to main text. (3) Added some text about threshold/optimum effects [reviewer's point that 'if a study uses two different forcing temperatures that both lie within the range of optimal forcing conditions, one would see no effect between the treatments and the authors would thus infer that forcing didn?t affect phenology, when in fact, forcing has a huge effect, not detected by the study design'] and our how our models try to handle this. [Can we also cite some figures in the supp for this? I thought this was an issue but I think we actually did see a effect most often.]
-great point!impossible to disentangel. much more info needed at species level. in absence of this...what is approach? one motivation for this paper is to highlight the need for additional work.
-we rely on the original researchers to separate forcing from chilling conditions- these were the treatments that they imposed. we therefore assume that they used a range of treatments that are relevant for their focal species. 
-the reviewer does not suggest an alternative approach....

-cite new schematic figure
- one of the findings of our paper- is that we need more work to accurately separate chilling and forcing
- the perceptive reviewer has highlihgted a general problem of the field...and yet...our models has been exrtemely predictive. 
Supplementary material 

\emph{p.2: What do you mean by `we included only studies with at least 49.5\% budburst?' This is not correct for most of the studies included in your OSPREE dataset. E.g., Heide (1993) and Zohner et al. (2016) defined budburst as the date when 1-3 buds on a twig had opened. Please clarify.}
\par We thank the reviewer for pointing out that we were not clear about our methods in the earlier version. What now have adjusted text on lines xx-xx to read as follows, "." As what we meant was [explain in greater detail, also ref our table about this from Cat! We actually did work on this and should clarify it in the text and for the reviewer.'

\emph {p. 3: Total chilling ranged from -1304 to 4724 Utah units? The Utah model allows for negative chilling units? What?s the biological justification for that?}
\par We agree with the reviewer that the Utah model, as well as other chilling models, are non-intuitive and biological justifications may not be immediately apparent. We now describe this model, as well as the other chilling model we used, Chill portions, on Lines XX. 
We could also add a little more discussion of this to supp.]
-the biological justification, as described by the developer of the original model (Richardson) is ...
\par We have also tried to make clear that in the main text that all chilling models are hypotheses. [
\par  We used the Utah model to report chilling because that allowed us to include the greatest amount of studies from the OSPREE database (i.e., because many studies used this model to estimate chilling). 

\par Models of how species accumulate chilling are poorly developed for forest trees, with few relevant tests evaluating the particular temperatures at which species do or do not accumulate chilling. Instead, researchers generally rely on models developed for perennial fruit trees (i.e., Utah (31) and chill portions (36), both of which were developed for peach species). These models are themselves hypotheses for
126 how chilling may accumulate and produce dormancy release, but are likely to be inaccurate for many species
127 (37).

\emph{p. 4: Latitude model: This model doesn't make sense to me. What is the latitude you refer to here? The location where the experiment took place? You refer to provenance locations, I doubt these are available for most of the studies, especially the ones conducted in botanical gardens or other collections.}

% EMW: ALERT! Ailene, this is semi-wrong and the reviewer is right in my mind. I just checked and we *do* enter provenance lat/long for Zohner but WE DO NOT HAVE THAT. We know where the trees were planted in Zohner (Munich botanical garden -- growing lat/lon), not where the seeds or material came from (provenance!). So this reply needs to be redone -- STEP 1 is to check all the studies used the latitude model are correct. After that we have to fix the database. I suspect that most studies entered by Tim or Dan are correct but we should check -- remember we have provenance lat/long and (AND!) growing lat/long. This an important difference and one we should check and fix. Let me know if this is not clear. 
% EMW: in general the changes are a great idea, but we need to check our work first and clarify up front to the reviewer that we know what he means and he is correct. It would be helpful to again cite our data and say we included both provenance and growing lat/long. 
%AKE: The model uses Provenance latitude. I agree that provenance has a certain connotation in regard to botanic gardens especially and this may be confusing. Therefore, I have adjusted my response, and the text in our manuscript, to try to avoid this language. Please let me know if this does not address your concerns, Lizzie, as I'm not sure that I entirely understand what you feel needs to be done.
\par We thank the reviewer for highlighting the need for greater clarity and methodological details about this model. In the initial submission, we did not adequately clarify what was meant by latitude. The latitude in our model refers to the latitude where plant material was growing at the time of collection, prior to being placed in experimental conditions. We referred to this as `provenance latitude' in our previous version of the manuscript, and realize now that this may have been confusing. For example, for the Zohner et al (2016) study, which was conducted at a botanical garden with specimen trees grown from source material , the `provenance latitude' refers to the location of the botanical garden where cuttings were collected from adult trees; we recognize now that the term  'provenance' in this context may be misleading, so have removed it from this new version, and instead have added details about the meaning of `latitude' in our model. For some studies, the latitude at which plant material was growing refers equivalently to the location where the experiment took place, as the reviewer suggests. This is not always the case, however; for other studies plant material was collected at one or more latitudes and then taken to a different latitude where the controlled chamber experiment was conducted. We have attached the full database for reviewers to investigate these details for each study included in our analyses; this database will be made freely available at the time of publication on KNB (). 
\begin{enumerate}
\item Lines XXX in the main text: we now say `While photoperiod had the smallest effect among the three cues, our results contrast with the extensive literature suggesting photoperiod is an unimportant cue for many species (Zohner et al 2016,K\"{o}rner \& Basler 2010)---instead we found it was surprisingly large, even when accounting for its interaction with latitude (i.e., the latitude from which plant material was collected prior to being placed into experimental conditions)
\item Supplemental Materials, Page 4: in the description of the Latitude Model, we now say: 'we examined the effect of including latitude in a model similar to our main one, but designed to estimate latitude effects. This model estimated the effects of each phenological cue (chilling, forcing, photoperiod) on days to budburst (as in the main model), in addition to the effect of collection latitude (i..e., the latitude from which plant material was collected prior to being placed in experimental conditions) and the interaction of photoperiod and latitude. We include this interaction because photoperiod effects are expected to vary by latitude...' and later in this paragraph we now say `..then subsetted the species and species complexes to include only those that had multiple collection locations across different latitudes.'
\item Supplemental Materials, Caption for Table S5: We have replaced 'latitude' with 'collection latitude' so that the caption now reads: `Using a model with Utah chilling units and testing the effects of collection latitude plus the interaction between latitude and photoperiod results in slightly muted effects...'
\item  Supplemental Materials, Caption for Figure S3: We have replaced 'latitude' with 'collection latitude' so that the title of the caption now reads: `Estimates for effects of chilling exceeded estimates for forcing, photoperiod, collection latitude, and the interaction between latitude and photoperiod, for most species...'
\end{enumerate}

\emph{Figures: Figs. 2 and 3, showing a 3-dimensional illustration of the interplay between winter chilling and spring warming, are very hard to read. I would prefer a simpler illustration.
}
\par We appreciate the Reviewer's perspective. We have shown both simpler 2-dimensional versions of these figures and the more complex 3-dimensional versions to a number of scientists and found opinions to be split on preferences for 2d versus 3d versions. To address this reviewer's concern, we have moved the 3-dimensional version of the manuscript to the supplemental materials, and now include 2 dimensional versions of Figure 2 and 3 in the main text.



Thoughts from Inaki:
- Maybe we should think less of chill being 3X more important than forcing but that this shows the hierarchy -- C >> F >> P.
- Sequential model means the ONLY dependency across parameters across models is that one phase must finish before the next.
- Overlapping or alternating or all parallel: are common other models -- Described in Chuine 2000 (Phenofit) paper -- they all say that amount of forcing depends on amount of chilling (alternating also covered in Pope et al. 2014). 
- Alternating shows up clearly in biological settings but is hard to explain at a physiological level. He thinks the 'growth competence' (aka parallel) model produces the same effect but is biological more realistic (it's in the Caffarra Betula papers). It suggests that the time to truly transition out of and fully release endodormancy can be fast or slow and depends on these factors. 


\boldtext{References cited}
11. J. Gauzere, et al., Agricultural and Forest Meteorology 244, 9 (2017).
12. K. Saikkonen, et al., Nature Climate Change 2, 239 (2012).
13.D. A. Way, R. A. Montgomery, Plant, Cell \& Environment 38, 1725 (2015).
\end{document}

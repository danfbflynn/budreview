\documentclass[11pt,letter]{article}
\usepackage[top=1.00in, bottom=1.0in, left=1.1in, right=1.1in]{geometry}
\usepackage{graphicx}
\usepackage{natbib}
\usepackage{amsmath}

\def\labelitemi{--}
\parindent=0pt

\begin{document}
%\bibliographystyle{/Users/Lizzie/Documents/EndnoteRelated/Bibtex/styles/besjournals}
 \bibliographystyle{/Users/aileneettinger/citations/Bibtex/styles/amnat.bst}
\renewcommand{\refname}{\CHead{}}


{\bf Titles}

Chilling dominates tree budburst in controlled climate experiments, but not in the great outdoors
Chilling outweighs photoperiod and forcing cues in temperate trees in experiments, but not in natural systems



\section{Figures, tables \& todo}

{\bf To do:}
\begin{itemize}
\item Estimate change in PEP sensitivities (based on our figures, use Cat's code -- I believe that's 1970-1980 and 2000-2010 -- do leafout date by mean spring temp)
note: Cat is checking GDD across two time periods.
\item Make new 2D and 3D figures for forecasting plots (Ailene)
\item Photoperiod x latitude effects figure to understand photo x lat interaction term  triptych (Ailene)
\item Experimental conditions forecasting figure (Ailene) {\bf Update:}  2D done- decided to try 3D)
\item Map of study locations, shading or symbol coding for number of cues (Lizzie)
\item Heat maps (Lizzie) {\bf Update:} These are close to done, need to make into one figure and tweak. 
\item PEPsims -- keep working on it ... next tasks: Look at Fu for GDD (does not) and think on what the temperature sensitivity actually is. (Fake data to examine hypothesis that constant cues with warmer springs could lead to less variability in leafout date, which could in turn lead to smaller estimates of temperature sensitivity.) -- Lizzie
\end{itemize}





\begin{enumerate}
\item $\mu$ plots
\item  $\mu$ forecasting figures: spring x winter warming -- PEP climate range and experimental climate range
\item Species forecasting with PEP data: \emph{Betula, Fagus} ... need to think on which ones to use (x sites x species focus etc). ... Maybe show photoperiod one?
\item PEP data figure with environmental conditions: as in Cat's figure + OSPREE data + maybe foercasting (at 2C or such?)
\end{enumerate}


{\bf Supplemental figures/tables:}
\begin{enumerate}
\item Map of study locations, shading or symbol coding for number of cues (Lizzie)
\item Map of species forecasting to justify sites
\item Tables, yes.
\item Heat maps for the main data, including by actual study design and by calculated chilling (our calculations)
\item Photoperiod x latitude effects figure
\end{enumerate}

\section{Outline so far...}

\begin{enumerate}
\item Why this matters ...climate change.
\begin{enumerate}
\item Climate change alters phenology
\item Increasing evidence of weakening sensitivities from long-term data... suggesting more than forcing is acting on spring phenology
\item We know three cues play a role...
\item But these cues are effectively imposible to estimate from long term data... A fundamental challenge of understanding the relative roles of these three cues is that, in the real world, they are often strongly correlated. 
\item Chamber experiments: their value --  often attempt to break this correlation to reveal mechanistic links between environmental conditions and budburst date. 
\item Chamber studies to date have constrasting effects (Zohner, Laube, Basler, Caffara etc.)
\end{enumerate}
\item Meta-analysis to the rescue!
\begin{enumerate}
\item Lit review of all woody plant spring phenology (total number of studies \& species, ref map, span of years)
\item Used only studies for which we could figure out forcing, photo and chill, chill we often calculated ourselves as so rarely reported (total number of studies, species in model)
\item To estimate the cues we used a Bayesian hierachical model 
\item This partially pools for a robust overall effect, and for robust effects for species with lots of data (\emph{Fagsyl, Betpen}) but pools towards the mean for species with fewer data ... (mention species complex here or put in caption)
\end{enumerate}
\item Short paragraph: Our results show that budburst phenology is determined by forcing, chilling, and photoperiod -- all three cues are important and all three advance budburst. Consistency across species -- fairly consistent with some variation in forcing, and then chilling, not much for photoperiod. 
\item Temperature, which is radically altered by climate change, was most important -- chilling and forcing show large effects on budburst
\begin{enumerate}
\item Chilling is the strongest strongset and most consistent cue (ref Laube and anyone else?)
\item Then forcing, consisent with experimental studies (CITES), observational (CITES)
\item Chilling is rarely manipulated directly, thus we had to calculate most of the chilling (impossible to provide estimates with only experimental chilling... ref supp heat maps)
\item Weinberger methods is most common for chilling and this is not a super way to measure it.
\item How you measure chilling matters a bit ... Utah vs. Chill portions
\end{enumerate}
\item Photoperiod
\begin{enumerate}
\item Photoperiod ... very consistent across species, suggesting all species do cue to it (ref Zohner, Caffara, Flynn ??)
\item The magnitude of photoperiod effects varies with latitude, with lower source latitudes generally having earlier budburst. Say provenance (population). 
\end{enumerate}
\item We did not estimate interactions. Why? ..  Add in node to process based models here. 
\begin{enumerate}
\item Very few studies actually design experiments to test for interactions, so there is little to build on
\item The few studies that do interactions often use the weinberger method, which seems a little weird based on our results.
\item They're hard for a couple reasons: need more reps, and photothermoperiodicity.
\item And! We cannot fully disentangle forcing vs. chilling conditions. (Chuine et al. 2016 GCB).
\item Our results average over interactive effects. 
\end{enumerate}
\item One paragraph: A simple interpretation of our model -- especially its chilling and photo effects -- predicts declining sensitivities in long-term data with climate change. This is because even though forcing increases, chilling is expected to decreases and photoperiods should get shorter -- both predicting delays, and thus an overall muted effect of temperature-only.  (Ref exp conditions forecasting figure.) But how do experimental temperature and photoperiod compare to predicted ones in nature? (Ref experimental conditions forecasting figure)
\item But how do the conditions overlap with natural conditions? (PEP + experimental data figures)
\begin{enumerate}
\item Forcing isn't bad
\item Experimental chilling is generally lower than field chilling
\item Photoperiod differences are very big in experiments
\item Declining sensitivities in PEP data (need to check)
\end{enumerate}
\item Forecasting with these semi-real data, however, do not predict a decline in sensitivity given the moderate amounts of warming already seen, instead they a suggest general advance of budburst until extremely high warming (ref. forecasting figure with PEP-based data)
\begin{enumerate}
\item Chilling often increases with small amounts of warming in some sites
\item Even if warming only happens in the winter, it takes a lot of warming to see a delay due to decreased chilling
\item At higher warming do see a leveling off or delay due to decreased chilling at some sites
\item Depends a lot on local climate... We also find that patterns of advancment with warming vary considerably depending on the current/background climate (e.g. how much advancement will continue with warming depends on how much chilling is currently experienced and whether that will increase or decrease with warming.)
\item (Compare advances in our models to PEP725 data?)
\item Photoperiod effects are minimal, even for \emph{Fagus}
\end{enumerate}
\item So why is PEP725 showing declining sensitivities?
\begin{enumerate}
\item Our results suggest few sites with delays before 3-4 degrees warming (CHECK)... and Germany has warmed X amount
\item Speeding up a biological process given sampling time resolution could lead to declining estimates of sensitivities, even if unchanged
\item Say something about what to do about this and how to figure out if this is the issue or it's cues. 
\end{enumerate}
\item Our results suggest most or all studied species are responsive to these three cues
\begin{enumerate}
\item Our results are only for one region, but highlight how critical accurate forecasts of shifts in forcing and chilling will be at local scales
\item To do this, we desperately need to better understand chilling (dormancy release) so that we can predict it in the future (maybe say need better models for chilling across species). 
\item Alongside this, we need more fundamental understanding of interactive cues, which requires larger studies across diverse species. Our results include these complexities but a finer understanding is needed in locations where cues do not change in concert.
\item These complexities are unlikely to alter our fundamental predictions of an increasing advance for many temperate trees in the future, even those with strong chilling or forcing cues (ref Gauzere) [Alt: An imrproved understanding of interactive cues, however, is unlikely to alter our fundamental predictions of an increasing advance for many temperate trees in the future, even those with strong chilling or forcing cues (ref Gauzere), unless cues are changing very asynchronously.]
\end{enumerate}
\end{enumerate}


\section{Reference notes}

EMW checked on \citet{fu2015} today and it is all done with declining sensitivities, calculated in days per degree, no evidence of GDD. I also noticed I have a back and forth with Yann about the paper when it first came out. So perhaps we should check with him? I also revisited `Impacts of global warming on phenology of spring leaf unfolding remain stable in the long run' which is also about days per degree sensititivies and they basically say you don't see in a moving-window approach, but that sort of makes sense. Anyway, see also Yann's email (which Lizzie is not posting as it is semi-personal).

AKE checked the following declining sensitivities citations:
\begin{enumerate} 
\item Chuine et al 2016 \citet{chuine2016} addresses process-based models that predict declining sensitivities to temperature
\item Piao et al 2017 \citet{piao2017} Examine relationships between spring land temperature and 2 metrics of start of spring (the spring zero-crossing date of atmospheric CO2 (SZC) and the magnitude of CO2 drawdown between May and June (SCC)). They find reduced sensitivity (strong relationship is first 17 years, gone in last 17 years) and speculate that it may be due to ``reduced chilling during dormancy and emerging light limitation"
\item gusewell et al 2017 \citet{gusewell2017} addresses ``shifts of the preseason" which they define as ``the most temperature?sensitive period preceding the phenological event." Definitely check this one out Lizzie!  ``Our results imply that declining temperature sensitivity can result directly from spring warming and does not necessarily indicate altered physiological responses or stronger constraints such as reduced winter chilling."
\item Vitasse et al 2018 \citet{vitasse2018} is related to the above, though it is not about declining sensitivities over time; rather with elevation (though the mechanism proposed is the same/similar to that in \citet{gusewell2017} ``we found that the elevation-induced phenological shift (EPS) has significantly declined....The stronger phenological advance at higher elevations, responsible for the reduction in EPS, is most likely to be connected to stronger warming during late spring as well as to warmer winter temperatures. Indeed, under similar spring temperatures, we found that the EPS was substantially reduced in years when the previous winter was warmer. Our results provide empirical evidence for a declining EPS over the last six decades. "
\end{enumerate}

\section{Reference list}

A few categories:\\

Papers about contrasting results over what cues matter from growth chamber studies: \cite{Basler:2012,Basler:2014aa,Caffarra:2011qf,Caffarra:2011a,Caffarra:2011b,Heide:2005aa,koerner2010b,Laube:2014a,vitasse2013,zohner2016}. Get Nanninga \emph{et al.} 2017: 'Increased exposure to chilling advances the time to budburst in North American tree species' and maybe Malyshev \emph{et al.} 2018 `Temporal photoperiod sensitivity and forcing requirements for budburst in temperate tree seedlings.'\\

Papers about declining sensitivities (Ailene will update this list): \cite{Rutishauser:2008fu,fu2015}. Also look for a Wang \emph{et al.} article `Impacts of global warming on phenology of spring leaf unfolding remain stable in the long run.' Vitasse paper on declining variation across elevation gradient. See \cite{yu2010}, but this is not temperate trees. \\

{\bf Papers about chilling units paper:} Fu 2012 from OSPREE. And these I know of ... \cite{harrington2015}\cite{lued2011,Luedeling:2011qe,Luedeling2013AgFM}. Note \citet{Harrington:2010} which includes a nice lit review of temperatures that matter \citep[][also has a good intro to dormancy]{harrington2015}. Then I did an ISI search (Harvard: TOPC: chilling model* AND Topic: utah* OR dynamic* AND Topic: phenolog*). Just doing this search stressed to me how much these models are for crops -- the percentage of titles including a fruit (I think peach and nectarine were maybe most common, followed by cherry or apple) was impressive. There are some mentions of Doug fir in the earlier records and clearly use of the models in other fields (like wild plant phenology) but all the papers that include words like `evaluating X model for Y species' come from crops ... so did the wild species folks ever verify these for wild species or is that basically impossible given the data required? \\

Anyway, here's my list from the ISI search: \cite{darby2016,Luedeling2015Acta,maulion2014,fu2014Plos,okie2011,anderson1992,Luedeling2009}\\

In particular, I think \citet{okie2011} has a good overview of the models (I think).\\

Another important paper to get (can someone send me a PDF) is \citet{Luedeling2015Acta} entitled`Chilling challenges in a warmer world.'

\bibliography{..//..//refs/ospreebibplus.bib}


\section{Miscellaneous bits}


\emph{Random text:} While climate change research benefits from this long history of study that has identified the proximate mechanisms of spring phenology \citep{chuinearees}, much current research still examines phenology from a simplified perspective. This is perhaps understandable as such complex cues are difficult to tease apart and most research has focused on a very few species and little to no information on all others. Whatever the reason however, the end result are studies that tend to use simplified metrics of phenology, especially in long-term observational studies---such as the simple calendar date when an event occurs or a correlation between budburst observed over time and a simple metric of temperature. This latter approach is often assumed to estimate `forcing' but must inherently integrate over other cues determining budburst each year---and the relative role of each cue likely varies year-to-year. 

Only old bit of outline it might be worth hanging onto (the rest seems covered in the current outline):

Species, population-level etc.
\begin{enumerate}
\item Latitude model
\item Provenance model
\item Trade-offs and correlations in cues among species 
\item Coastal versus non-coastal
\item Continent of origin?
\end{enumerate}


\end{document}


\begin{enumerate}
\end{enumerate}
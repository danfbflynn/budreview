\documentclass[11pt,letter]{article}
\usepackage[top=1.00in, bottom=1.0in, left=1.1in, right=1.1in]{geometry}
\usepackage{graphicx}
\usepackage{natbib}
\usepackage{amsmath}

\def\labelitemi{--}
\parindent=0pt

\begin{document}
\bibliographystyle{/Users/Lizzie/Documents/EndnoteRelated/Bibtex/styles/besjournals}
\renewcommand{\refname}{\CHead{}}

{\bf To do:} Fake data to examine hypothesis that constant cues with warmer springs could lead to less variability in leafout date, which could in turn lead to smaller estimates of temperature sensitivity.

{\bf Titles}

Chilling dominates tree budburst in controlled climate experiments, but not in the great outdoors
Chilling outweighs photoperiod and forcing cues in temperate trees in experiments, but not in natural systems

\section{Outline}

{\bf Figures}

\begin{enumerate}
\item $\mu$ plots
\item Forecasting figures: spring x winter warming
\item Species forecasting?
\item Map of study locations?
\end{enumerate}

{\bf Should incude ....}
\begin{itemize}
\item How you measure chilling matters a bit
\item Most species appear to respond to all cues
\item Species vary in their cues
\item Latitude matters, latitude matters to photoperiod
\item We did not estimate interactions. Why?
\begin{itemize}
\item Very few studies actually do, so there is little to build on
\item The few studies that do interactions often use the weinberger method, which seems a little weird based on our results.
\item They're hard.
\item Our results average over main effects. 
\end{itemize}
\item So why is PEP725 showing declining sensitivities?
\item What are the next steps?
\begin{itemize}
\item We need more studies on interactive cues
\item We desperately need to better understand chilling
\end{itemize}
\end{itemize}

{\bf Outline so far ...}

\begin{enumerate}
\item Why this matters ...climate change,, 
\begin{enumerate}
\item Climate change alters phenology
\item Increasing evidence of weakening sensitivities
\item We know three cues play a role...
\item but effectively imposible to estimate from long term data
\item Studies to date appear constrasting (Zohner, Laube etc.)
\item Meta-analysis to the rescue!
\end{enumerate}
\item Chilling has the biggest effect, then forcing, then photoperiod
\item We do some curde forecasting (explain how)
\item Forecasting implications
\begin{enumerate}
\item Chilling actually increases in some areas with small warming
\item Even if warming only happens in the winter, it takes a lot of warming to see a delay 
\item Photoperiod effects are minimal, even for \emph{Fagus}
\item Location matters to when chilling does decine
\end{enumerate}
\end{enumerate}


\end{document}

\begin{enumerate}
\end{enumerate}
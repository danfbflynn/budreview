\documentclass[11.5pt,a4paper]{letter}
\usepackage[top=.75in, bottom=.75in, left=1in, right=1in]{geometry}
\usepackage{graphicx}
\begin{footnotesize}
\address{1300 Centre Street \\ Boston, MA, 20131}
\end{footnotesize}
\begin{document}
\begin{letter}{}
\includegraphics[width=0.3\textwidth]{/Users/aileneettinger/Dropbox/Documents/Work/AA_heading.pdf}
\pagenumbering{gobble}

\opening{Dear Dr. Findlay:}
Please consider our paper, entitled ``Winter temperatures dominate spring phenological responses to warming'' for publication as a `Letter' in \emph{Nature Climate Change}. This manuscript is a revised version of an earlier submission (NCLIM-19081773). We include a point-by-point response to reviewer comments. 

\par As you may recall, our manuscript utilizes a new global database to address a research topic of critical relevance to a broad reach of \emph{Nature Climate Change} readers:  the timing of spring phenology (e.g., budburst) in woody plants. Spring phenology impacts plant fitness, shapes plant and animal communities, affects wide-ranging ecosystem services from crop productivity to carbon sequestration and unites the fields of biometeorology, ecology, cellular and molecular biology. Our work is groundbreaking in its synthesis of four decades of research across 72 experiments to quantify the relative importance of three environmental cues critical to phenology for 203 species from around the globe. 

\par The three reviewers recognized the potential of our work to influence future research. They also highlighted some concerns in regards to our methods, how we presented those methods, and how we addressed the complexity of the topic. Given these concerns we have substantially revised our manuscript. Our original results and conclusions are unchanged but we believe our revised manuscript is much stronger and more clearly shows how and why our results are robust.

\par Our manuscript contains new and revised main text figures, a substantially revised text from the abstract to conclusions sections, and several new analyses (and related new tables in the supplement). For example, we have added a new model that tests for effects of life stage and applied a recently published modelling approach for estimating temperature sensitivity (sliding windows), as suggested by reviewers. We have also created a new figure (Figure 1) that at once better explains our methods and highlights the complexity of estimating forcing and chilling effects in experiments. We also have modified previous figures in the main text to address reviewer concerns. Finally, we also have added substantially to the online `Methods' section, which adheres to the new guidelines of \emph{Nature Climate Change}. 

\par Upon acceptance for publication, the database will be freely available at KNB (currently meta-data are there); the full database is available to reviewers and editors upon request. This work is a meta-analysis, so data have been previously published; however, the synthesis of these data and the tables, figures, models, and materials presented in this manuscript have not been previously published nor are they under consideration for publication elsewhere.

Sincerely,\\

\includegraphics[scale=.4]{/Users/aileneettinger/Dropbox/Documents/Work/AileneEttingerSignature.png} \\
Ailene Ettinger\\
\begin{footnotesize}
Quantitative Ecologist, The Nature Conservancy- Washington Field Office\\
Visiting Fellow, Arnold Arboretum of Harvard University 
\end{footnotesize}

\end{letter}
\end{document}

\documentclass[11pt,a4paper]{letter}
\usepackage[top=1.00in, bottom=1.0in, left=1.0in, right=1.0in]{geometry}
\usepackage{graphicx}
\usepackage{natbib}
\usepackage{gensymb}
\address{1300 Centre Street \\ Boston, MA, 20131}
\begin{document}
\bibliographystyle{/Users/aileneettinger/citations/Bibtex/styles/nature.bst}

\begin{letter}{}
\includegraphics[width=0.3\textwidth]{/Users/aileneettinger/Dropbox/Documents/Work/AA_heading.pdf}
\pagenumbering{gobble}



\opening{Dear Editors:}
Please consider our paper, entitled  ``Spatial and temporal shifts in photoperiod with climate change'' as a Review \& Synthesis in \emph{Ecology Letters}. This manuscript is a revised version of manuscript  ELE-01441-2019. We have incorporated the suggestions of the referees and editor, as detailed in the enclosed point-by-point response. 

As you may remember, our paper addresses an urgent ecological question: What are the implications of the altered photoperiod that plants and animals experience as they shift their ranges and seasonal activities with climate change? The two most-observed biological impacts of climate change are shifts in space (range shifts) and time (phenological shifts). Both alter experienced photoperiod, which could dramatically affect performance and fitness. However, the magnitude of effects from shifts in photoperiod with climate change are unknown or unquantified for the vast majority of species.  We address this need by synthesizing the large body of controlled climate experiments that test effects of temperature and photoperiod on spring phenology in plants \emph{(1)}. Our review differs from previous work \emph{(e.g., 2)} in that it quantifies expected changes in experienced photoperiod due to shifts in space versus time. 
% EMW -- We don't need to review the issues in such depth, we need to say whatweyou changed and stay focused on that. I tried to remember what changed, but please check my work here and edit as you see fit.
\par All three reviewers felt our review addressed an interesting and important topic, and all also made suggestions for improvement. To address their concerns we have: restructured the introduction to focus first on the role of photoperiod, added text and references describing the critical importance of photoperiod in spring phenology, added text and references regarding animals to underscore the broad relevance of this issue beyond woody plants, and integrated text on suggested improvements to experiments. We have further adjusted the figures and legends for greater clarity. Finally, we have added a new box `Dominant models of how photoperiod affects spring woody plant phenology' to highlight gaps that new physiological experiments could best fill. We have aimed to address all of the reviewer concerns, though we have stuggled to address some of the concerns of Reviewer 1, given our reading of the current literature. For example, we believe photoperiod is relevant to spring phenology, given the many recent studies demonstrate that it is a critical cue affecting spring budburst in plants, the timing of migration in birds, and many other biological events. We have added new references and text that support our focus on spring phenology. Please see our detailed, point-by-point response below for more information on our changes.
% Reviewer 1 felt that the overall topic and some figures were poorly explained. Reviewer 2 noted that the previous version of the manuscript was mostly plant (and tree) oriented and suggested additional emphasis on the relevance to animals. Reviewer 3 felt that we should more fully discuss the need for improvements to experiments, and specifically that the need for additional plant physiological experiments should be emphasized. 
% I would save this language for the editor, and not include it here: While we have aimed to address all of the reviewer concerns, we do feel there is a discrepancy between some of the statements made by R1 and the current literature. For example, Reviewer 1 states that  photoperiod is not important for spring phenology; however, many recent studies demonstrate that it is a critical cue affecting spring budburst in plants, the timing of migration in birds, and many other biological events. We have added new references and text that support our focus on spring phenology.
\par  We believe the new manuscript is much improved and hope you will find it suitable for publication in \emph{Ecology Letters}.


Sincerely,\\

\includegraphics[scale=.4]{/Users/aileneettinger/Dropbox/Documents/Work/AileneEttingerSignature.png} \\
Ailene Ettinger
\begin{footnotesize}\\
Quantitative Ecologist, The Nature Conservancy- Washington Field Office\\
Visiting Fellow, Arnold Arboretum of Harvard University 
\end{footnotesize}
\\
\\
\noindent \emph{References in cover letter}

\begin{footnotesize}
\begin{enumerate}
\item Wolkovich, E.M.,  et al. 2019. Observed Spring Phenology Responses in Experimental Environments (OSPREE). Knowledge Network for Biocomplexity. urn:uuid:b2ab2746-b830-4
\item Saikkonen, K., et al. 2012. Climate change-driven species' range shifts filtered by photoperiodism. \emph{Nature Climate Change}, 2:239.
\end{footnotesize}
\end{letter}


\underline{\subtitle{Response to Reviewers}}

\emph{Reviewer Comments are in italics.} Author responses are in plain text.\\

 \emph{\bold{Reviewer \#1 (Comments for the Authors)}}\\

\emph{I am happy to review this manuscript as a Reviews and Synthesis paper to Ecology Letters. Basically, this paper aims at answering the three questions: 1. How will climate change alter the photoperiod experienced by organisms? 2. What are the implications of altered photoperiods for biological responses to climate change? 3. Can research apply data from experiments that alter photoperiod to aid in forecasting biological implications of climate change? Absolutely, these three questions are very important for understanding the changes of photoperiods and its impacts to biological phenophases. However, it is pity that I found very limited new knowledge by reading this paper, and many places are very difficulty to follow. I have several major concerns as followings. }


\par We thank the reviewer for taking the time to review our paper and provide detailed comments. We appreciate that the reviewer agrees the questions on which we focus in our paper are very important. We have modified the manuscript to address the reviewer's concerns. In particular, we have worked to more clearly articulate the goals of our paper. As a Review paper, this manuscript does not present new data; rather, the primary goals of this manuscript are to synthesize previous studies, call attention to new research needs, and suggest novel approaches for future work. For example, in the abstract we have modified the text to clarify that our work is a review in the following places:

\begin{itemize}
\item Line 4, we have added ``of reviewed studies" so that the sentence now says: ``As photoperiod is a common trigger of seasonal biological responses (affecting plant phenology in 84\% of reviewed studies that manipulated photoperiod), shifts in experienced photoperiod may have important implications for future distributions and fitness.''
\item Line 7, we have added ``we synthesize published studies" so that the phrase says: ``Here we synthesize published studies to show that impacts on experienced photoperiod from temporal shifts."
\end{itemize}

\par \emph{1. The authors seem to focus on the responses of spring phenophases to photoperiods, but it may be a wrong direction. There is obvious asymmetry of radiation and temperature on seasonal changes. The peak of radiation is always earlier than that of temperature, and which means the limited environmental variable in Spring is temperature not radiation. Therefore, the photoperiod is not important for plant phenology in Spring. On contrary, in autumn, the radiation or daylength is very low when the temperature keeps still very high. However, I did not find some reviews on the changes in autumn phenophases in response to photoperiod.}\\

\par The reviewer is correct that our work focuses on spring, not autumn, phenology. However, we focused on spring phenology in part because of the large body of existing scientific literature documenting that photoperiod plays a critical role in determining plant phenology in the spring. We now restructured the introduction, as Reviewer 1 suggested in item 3, and now review the relevance of photoperiod in spring phenology in Lines 25-35, where we write: 
``For many organisms, the timing of spring events-- i.e., phenology, including  flowering, bird arrival, egg hatching and myriad other biological activities-- is thought to be determined by photoperiod interactively with temperature (Fu et al., 2019; Winkler et al., 2014, see also Box 1).  The strong role of temperature is apparent in recent advances in spring phenology, some of the most widely documented signals of climate change. At a given location on Earth, annual patterns in photoperiod have not changed as climates have warmed. Yet, across taxa, from plants and insects to mollusks and mammals, spring phenology is occurring earlier as temperatures warm, with average shifts of 1.2 to 5.1 days earlier per decade (Bradley et al., 1999; Parmesan36and  Yohe,  2003;  Poloczanska  et  al.,  2013;  Root  et  al.,  2003)  or  1.3  to  5.6  days  earlier  per \degree C  of  warming (Polgar et al., 2013; Wolkovich et al., 2012).  These changes are some of the largest climate change-induced biological shifts observed, with early spring phenology shifting more rapidly than later season phenology in most cases (Bradley et al., 1999; Menzel et al., 2006).''
\par We have also added text explaining our choice to focus on spring phenology to the Introduction Lines 69-72, where  we write:
``We focus on spring events, as phenology during this time is one of the most widely observed and rapidly changing biological responses to climate change (Parmesan, 2006). In addition, the role of photoperiod is less understood in spring phenology compared with autumn phenophases, but recent studies showing declines in responses of spring budburst to warming (e.g., Fu et al., 2019; G\"{u}sewell et al., 2017; Yu et al., 2010)  suggest that photoperiod constraints may be imminent.''

\par We chose to focus on the intersection of photoperiod with spring phenology in part because, the topic has received growing interest (e.g., Chamberlain et al 2019, Fu et al 2019, Richardson et a 2018), given recent studies highlighting declining responses of spring budburst to warming (e.g., Fu et al 2019, G\"{u}sewell et al 2017, Yu et al 2010). One often-hypothesized cause of these declines in sensitivity to warming is photoperiod thresholds. Though photoperiod cues dominate in the fall for many organisms, as the reviewer suggests, fall phenology responses to climate change have been muted. In contrast, spring phenology responds strongly to temperature and thus has advanced substantially with warming---causing cascading, and generally unexplored, effects on photoperiod experienced at the start of spring. Thus, we wish to shine a light on the potential implications to spring phenology of altered photoperiod cues with climate change. We also suggest that there are many opportunities for additional research in this field, given the wealth of data from growth chamber experiments on woody plant spring phenology. 

\par In addition, we focus on the intersection of photoperiod with spring phenology because photoperiod has received far more detailed attention for end-of-season activities, such as growth cessation in the fall, than for spring activities. Indeed, there are already several reviews covering this topic (e.g., Azeez  \& Sane 2015,  Gallinat et al 2015, Lagercrantz 2009, Allona et al 2008).
% EMW: If we have cited any of these papers now in the main text, we should mention it... "Indeed, there are already several reviews covering this topic (e.g., Azeez  \& Sane 2015,  Gallinat et al 2015, Lagercrantz 2009, Allona et al 2008), which we now note on lines XX-XX."
%Nacho says: Should we add some brief mention to data availability? I know that should not be the main motivation for the paper, but if there are not as much data available on growth chamber experiments looking at fall phenology (perhaps there are and I'm not aware of it), then that would preclude us from doing what the reviewer suggests.

\par \emph{2. the abstract is very week because it fails to summarize the main content of this review manuscript. The authors are going to review the study progresses on photoperiods and its impacts, therefore, the abstract should also focus on the three key scientific questions. However, the current abstract does not highlight the key results and conclusions but repeat the background. }

\par We have modified the abstract, so that it includes a sentence addressing each of the three key scientific questions. (Due to word limits, we did not add the questions themselves). 
\par To address question 1 (``How will climate change alter the photoperiod experienced by organisms?") the abstract includes the following sentence (Lines 7-9):
``We synthesize published studies to show that impacts on experienced photoperiod from temporal shifts could be orders of magnitude larger than from spatial shifts (1.6 hours of change for expected temporal versus only one minute for latitudinal shifts)."
\par To address question  2 (``What are the implications of altered photoperiods for biological responses to climate change?"), we have added the following sentence (Lines 12-13 )```For example, growth chamber experiments on woody plant spring phenology often have data relevant for climate change impacts, and suggest that shifts in experienced photoperiod may increasingly constrain responses to additional warming.''
\par To address question  3 (``Can research apply data from experiments that alter photoperiod to aid in forecasting biological implications of climate change?"), the abstract states the following (Lines 10-11):  ``Incorporating these effects into forecasts is possible by leveraging existing experimental data; for example, growth chamber experiments on woody plant spring phenology often have data relevant for climate change impacts."

\par \emph{3. The introduction starts from the impacts of climate warming on phenophases (the first paragraph). I would like to read the importance of photoperiods first, and then to know the connections between temperature and photoperiods. }\\
\par  We thank the reviewer for this suggestion, which we feel helps the introduction better emphasize the central topic of our paper. We have restructured the beginning of the introduction as the reviewer suggests, so that the paper now begins with a paragraph on photoperiod (Lines 18-24), followed by a paragraph on interactive effects of temperature and photoperiod (Lines 25-35). \\


\par \emph{4. Line 36: these two references did not support your statement "Some studies suggest that, with additional warming, photoperiod will limit phenological shifts of certain species such that they will not track rising temperatures (e.g., by leaf out earlier in the spring, K\"{o}rner and Basler, 2010; Way and Montgomery, 2015)." These two papers are review papers, and both describe the contrary conclusion with the authors. For example, the first paper states "However, no study has demonstrated that photoperiod is more dominant than temperature when predicting leaf senescence (1), leafing, or flowering" The authors should double check and be cautious for all conclusions in the manuscript. Therefore, I doubt the main topic in this review paper as I mentioned before that the impacts of photoperiods is very limited to phenophases in Spring.} 
\\
\par We agree with the reviewer that our characterization of the papers as ``studies" may have been misleading, as the referenced articles are not experiments, and we thank the reviewer for pointing this out. One is review paper summarizing many observational and experimental studies (Way and Montogmery 2015). The other (Basler and K\"{o}rner 2010b) is a response to a response to a perspective (Basler and K\"{o}rner 2010a). It was not our intention to be misleading. To address this, we have removed the original K\"{o}rner and Basler citation (K\"{o}rner and Basler 2010b), and added instead references to a long-term observational study (Fu et al 2015), growth chamber/common garden experiments (Fudickar et al 2016, Basler \& K\"{o}rner 2012), and a perspective (K\"{o}rner, C. \& Basler 2010a), in addition to the review paper we originally cited (Way and Montgomery 2015). We have also modified the text slightly. The new phrase says:
``It has been suggested that, with additional climate change, photoperiod will limit phenological shifts of certain species such that they will not track rising temperatures (Fudickar  et al 2016, Fu et al 2015, Way and Montgomery 2015, Basler \& K\"{o}rner 2012, K\"{o}rner \& Basler 2010)."

% I think this is a good reply, but have made it sound friendlier. 
\par Beyond this, we have double-checked all our citations and confirmed they are correct. We note that the quote by the reviewer ``However, no study has demonstrated that photoperiod is more dominant than temperature when predicting leaf senescence'' is not from to K\"{o}rner and Basler (2010b)  as the reviewer suggests. This quote is from a ``Response" to K\"{o}rner and Basler (2010a), written by Isabelle Chuine, Xavier Morin, and Harald Bugmann (Chuine \emph{et al.} 2010). Chuine \emph{et al.} (2010) make important points in their response to K\"{o}rner and Basler (2010a), but we did not reference their writing in our manuscript. Further, we have worked to make sure that we have misquoted the any references we cit. In considering the two references the reviewer was concerned about (K\"{o}rner and Basler, 2010b; Way and Montgomery, 2015), these are examples of writing that describes the idea that, with future warming, photoperiod cues may limit plant responses to temperature. To clarify this, we have provided below some direct quotes from these two citations: % We encourage the reviewer and editor to consult the references directly to confirm the accuracy of our summary of the authors' statements:
\begin{itemize}
\item From K\"{o}rner and Basler (2010b): ``Warm temperatures are most effective in promoting budburst only after the photoperiod requirement of adequately chilled buds is met (3, 6). This process prevents trees from sprouting before the risk of freezing damage is over. Photoperiodism in trees has been known for almost 100 years (7, 8), but is an often ignored environmental cue when predicting the effects of a future climate."
\item From Way and Montgomery (2015): ``Here, we discuss how day length may limit the ability of tree species to respond to climate warming in situ, focusing on the implications
of photoperiodic sensing for extending the growing season and affecting plant phenology and growth, as well as the potential role of photoperiod in controlling carbon uptake and water fluxes in forests."
\item From Way and Montgomery (2015): ``Organisms that rely on photoperiodic cues for sensing the arrival of spring and the approach of winter may have constrained responses to warming, which would limit expected extensions of the growing season in mid- and high latitude forests (Saikkonen et al. 2012).
\item From Way and Montgomery (2015): ``There is also evidence for direct photoperiodic sensitivity of bud burst in some species (e.g. F. sylvatica; Heide 1993a): at shorter photoperiods, it takes longer for buds to burst than at long photoperiods, indicating that long photoperiods enhance dormancy release (Heide 1993a,b). Tree species that fall into this category are likely to be strongly limited in their ability to respond to warmer springs by breaking bud earlier."
\item From Way and Montgomery (2015): ``A reliance on photoperiod as a seasonal cue for growth and physiological activity may either constrain forest productivity,
via limitations on photosynthetic capacity, migration potential or growing season length, or permit longer growing seasons and greater tree growth in individuals that migrate to higher latitudes."
\end{itemize}

\\
\par \emph{5. Figure 1 and 2 are two the most important results in this review paper. However, the authors fail to introduce their importance clearly. Figure 1 only shows the temporal changes of daylength at two locations, and I have no idea about other information in the figure. Figure 2 is very low level greenness map, and I did not find any useful information at all. In addition, these two figures can not support the conclusion in the text very well. For example, how can the authors arrive at this conclusion "A general pattern of longer photoperiod at green-up toward the poles is consistent across years (Fig. 2b) and green-up does not appear to occur at daylengths less than 10 hours."? (Line 82-83)}
\par We thank the reviewer for pointing out that our explanations of the figures were unclear in the previous version. In this new version, we have modified the figures and adjusted the figure legends to improve clarity, as follows:

\par We have modified Figure 1 and its legend to address the reviewer's comments.  We have added text to Figure 1 to make clear the patterns how photoperiod varies with latitude, and that potential spatial and temporal shifts in photoperiod with climate change vary seasonally and with latitude. For example, we have added units to figure to make clear what is being shown in each inset (e.g., ``minutes" of change in daylength). Figure 1 legend now says: ``Temporal shifts in activity yield larger changes in experienced photoperiod compared to spatial (latitudinal) shifts} on the same day of year, due to patterns in photoperiod variation with latitude and by day of year. Here, we show this variation at two latitudes (22.5\degree, 45\degree), using hypothetical spatial and temporal shifts. These shifts are based on observed rates with recent global warming: 6-17 kilometers per decade, or approximately 0.5-1.5 degrees in 100 years, for spatial shifts (Parmesan and Yohe 2003,Parmesan 2006}, and 2-3 days per decade, or 30 days in 100 years, for temporal shifts (Parmesan 2006, Chen et al 2011}). They highlight the greater magnitude in daylength changes in the early spring, close to the vernal equinox (e.g., day of year 91), versus close to the summer solstice (e.g., day of year 182)."

\par To further clarify the idea of a `temporal shift' versus a `spatial shift' in experienced photoperiods, we more clearly define these ideas in the abstract. We have now added some defining phrases to Lines 1-2: ``Climate change causes both temporal (e.g., advancing spring phenology) and geographic shifts (e.g., range expansion poleward) in species; these shifts affect the daylength (photoperiod) experienced.
\\
\par We have modified Figure 2 and its legend to address the reviewer's comments. We have added ``hours" to the color ramp legends in each panel to clarify that what is shown is the daylength or change in daylength at greenup. Figure 2 supports the text in the manuscript because it is  a useful proof of concept for to demonstrate that 
\begin{itemize}
\item Spatial variation in photoperiod is consistent across years at large scales (at near-polar latitudes, greenup only occurs under maximum photoperiod). This may appear to be trivial, but we are not aware of other publications synthethizing this idea in a single figure.
\item At finer spatial scales, photoperiod shifts inter-annually may be highly heterogeneous, even at a given latitude (up to 5 hours of variation). In Figure 2C, for example, look at latitude 32.7 in North America, areas close to the west coast (near San Diego, California, USA and Tijuana, Mexico); this region experienced much longer photoperiods at green-up (+3 hours, the green areas of Fig 2C) in 2012 (a year of early greenup across most of North America) compared to 2009 (an average year). Further inland/east, however, the photoperiod at green-up was two hours (purple areas in Fig 2C)  less in 2012 compared with 2009, at latitude 32.7.

\item There are consistent daylight thresholds below which greenup does not seem to happen (10 hours at least in North America and Western Europe), as shown by the lowest values in Fig. 2A,B.
\end{itemize}
\par We have added text clarifying the ways that Figure 2 supports our statements in the main text by adding text to more clearly walk the reader through the above points in Lines 83-93, where we write:
\par ``Some consistent latitudinal patterns in experienced photoperiod are apparent at a braod scale.  For example, the pattern of longer photoperiod at green-up toward the poles is consistent across years (i.e., on the day of year when green-up occurs close to the north pole, daylength approaches 24 hours in both an ``average" year, Fig.  2A, and in an ``early" year, Fig.  2B). Note that green-up does not appear to occur at daylengths less than 10 hours, across North America and Europe."
\par ``Despite these consistent patterns at a broad scale, there is also strong spatiotemporal variation in experienced photoperiod across years.  Compare the photoperiod at green-up in an ``early" versus an ``average" year (Fig.1092): experienced photoperiod at green-up can vary by two to three hours from one year to the next in the same location (Fig.  2C).  We use green-up date as an example here because it is an available dataset and represents an important biological event, signalling the start of the growing season. Though green-up date corresponds to plant phenology, we expect that spatiotemporal patterns are similarly hetergeneous in spring
phenology of other organisms (Ovaskainen et al., 2013; Penuelas et Penuelas et al., 2002).''
\par ``Against this existing background variation,  climate change will cause shifts in experienced photoperiod as species respond to warming temperatures.  Spatial shifts in species' ranges and temporal shifts in phenology will alter the photoperiods experienced by organisms with future climate change.  The magnitude of these alterations will vary depending on the organism's location and the type of shift(s) it undergoes.  For example, poleward shifts in species' ranges cause organisms to experience a wider range of daylength throughout the year (Fig.  1).  Elevational shifts, in contrast, cause minimal changes in the range of daylength throughout the year."
% EMW: This seems all very good! Nice work.

\par \emph{6. the authors fail to summarize the knowledge on "How will climate change alter the photoperiod experienced by organisms?" I only get very limited knowledge on current conditions of photoperiods experiences by organisms and future changes. In addition, these information focus on the local scale, no global pattern shown. I do not think this kind review can promote science advances. Same feelings for other two sections.} 
\\
\par We agree with the reviewer that there is a great need for more global patterns in experienced photoperiod. We present global patterns in experienced photoperiod at green-up date in Figure 2. We use green-up date as an example because it is an available dataset and represents an important biological event that signals the start of the growing season and affects the growing season length and therefore carbon sequestration.  We now state this explicitly on Lines 90-93 where we write:
``We use green-up date as an example here because it is an available dataset and represents an important biological event, signalling the start of the growing season. Though green-up date corresponds to plant phenology, we expect that spatiotemporal patterns are similarly heterogeneous in spring phenology of other organisms (Ovaskainen et al., 2013; Pe\~nuelas et al., 2002).''
We welcome specific ideas for additional global phenology datasets available to use as examples, or details on what specific additional information the reviewer would like to see included. Indeed, part of our goal in writing this manuscript is to highlight the need for additional research attention in this area.
%Nacho: I'd emphasize fig 5, and perhaps reference it in that subsection of the text (then it would become fig 3). In addition, we can offer the editor to extract photoperiod data for phenofit studies, but noting that even if we did that, it would still be for a biased subset of species, which falls beyond the scope of our paper. We can point out that it would be interesting to see future research documenting projected shifts in photoperiod for a large enough set of species and taxa.

\par \emph{7. Figure 3 and 5 are results derived from OSPREE database, but it is very hard to understand how the experiments have been conducted and if the conclusions are reliable because the authors did not introduce them adequately.} 
\\
\par Thank you for sharing the need for a more full explanation of the experiments in our OSPREE database. We have added a full description of OSPREE to Box 1, which we now cite each time we mention OSPREE in the main text or figure legends. This box now provides some details on the OSPREE database, as well as references for where additional  details can be found. We have also added methodological details to KNB, where the OSPREE database will be available at the time of publication \url{https://knb.ecoinformatics.org/view/urn:uuid:b2ab2746-b830-436b-a7a9-01b3ef3558e4}. % (The OSPREE database is available to reviewers upon request as well.)

\par \emph{8. There are too many conclusions no evidence supported to list them all.}
\par We have added additional references to other studies, in addition to referencing our own figures, to support our conclusions. We welcome more detailed feedback on what specific statements, if any, are in need of additional evidence. 

\\
\par \emph{\bold{Reviewer \#2 (Comments for the Authors)}} % EMW: When I read this before I thought it seemed good, so I did not re-review it.
\\
\par \emph{Manuscript is a good review of what is known about the effects of photoperiod in the context of climate change. Although the title doesn't make this clear, it's mostly plant (and tree) oriented.  I think I made a few notes about places where relevance to animals could be included.  It's pretty well written, although I've flagged some editorial issues on the PDF.  I think the relevant literature is cited, and the figures are useful and clear.  The ideas are presented clearly and the discussion is good.  See a few other comments on the PDF. } 
\\
\par We thank the reviewer for these comments. We have incorporated the relevance to animals in places where the reviewer suggested  (e.g., by mentioning the affect of photoperiod on reproduction in animals, and by adding additional references focused on photoperiod effects on animals). We feel this has strengthened the manuscript by demonstrating the broad range of taxa affected by photoperiod. We have also addressed the editorial issues noted in the pdf, which are summarized below:
\\
\par \emph{Do  you mean latitudinal shifts?}
\par Yes! We have replaced ``spatial'' with ``latitudinal'' to be more specific. The phrase now says ``(1.6 hours of change for expected temporal shifts versus one minute for latitudinal shifts)."
\\
\par Throughout: We replaced "climate change induced" with ``climate change-induced" as suggested by the reviewer.
\par Line 42: \emph{cues? The cues won't change. maybe 'responses'?}
\par We thank the reviewer for pointing out a lack of clarity here. We have rewritten the phrase, as suggested, which now reads: ``The extent to which daylength constrains phenology will depend in part on how rapidly photoperiod responses can acclimate or adapt to new environmental conditions, which remains poorly understood (Grevstad and Coop, 2015)."}
\par Line 55-56 \emph{Do you mean this to include reproduction? If not, add reproduction to that list. }
\par We have added reproduction to the list, which now reads: ``...since daylength can affect the timing of development (Grevstad and Coop, 2015; Muir et al., 1994), migration
(Dawbin, 1966), reproduction (Dunn, 2019; Dardente, 2012; Ben-David, 1997), and other important responses. "
\par Line 66: We have replaced ``research" with ``researchers" as suggested by the reviewer.
\par Line 119: We have replaced ``if" with ``whether" as suggested by the reviewer.
\par Line 125-131: We have added reproduction to the list of animal responses affected by daylength, as suggested by the reviewer. This section now says:
``Daylength, often in combination with temperature, can play a role in controlling critical biological functions, including vegetative growth, cell elongation, budburst, and flowering in plants (Heide and Sonsteby, 2012; Heide, 2011; Hsu et al., 2011; Sidaway-Lee et al., 2010; Mimura and Aitken, 2007; Linkosalo and Lechowicz, 2006; Erwin, 1998; Ashby et al., 1962) and growth rate, maturation, reproduction, migration, and diapause in animals (Dunn, 2019; Zydlewski et al.,2014; Dardente, 2012; Tobin et al., 2008; Bradshaw and Holzapfel, 2006; Ben-David, 1997; Muir et al., 1994; Saunders and Henderson, 1970; Dawbin, 1966)."
\par Lines 163-164: We have added hyphens, as suggested, so that the phrase now reads: ``slower-growing or later-emerging ones"

\par Line 190: We have replaced ``if" with ``whether", as suggested.
\par Line 208: We have added a hyphen, as suggested, changing ``fine scale" to ``fine-scale". 
\par Lines 245-269 (Glossary): We have ensured that all definitions end with periods, to be consistent, as suggested by the reviewer. 
\par Line 317: We have made the suggested change to our wording. The phrase now reads: ``..some species demonstrated a response to photoperiod opposite to that typically observed..."

\par References:  We have added the publisher of the book, as suggested by the reviewer. Thank you.
\par References: We have capitalized the genus names, and italicized the genus, species, and variety names throughout. Thank you for noticing this error.

\par Figure 1: We have replaced ``Spatial shift" with ``Latitudinal shift" in the figure legend, as suggested.

\\
\
\newline
\par \emph{\bold{Reviewer \#3 (Comments for the Authors)}}
\\
\par \emph{The shifts of photoperiod is not only the result of phenology changes, such as earlier in spring leaf-out and geographic shifts in species distribution, but also the reason/driver of phenology change, for example the shortening photoperiod associating with earlier leaf-out reduced the temperature sensitivity. I do think this is an interesting review in the photoperiod shifts, although many papers have discussed such photoperiod effect on phenology, and agree that the photoperiod effect should be coupled into the phenology modeling, especially considering the larger uncertainty of phenology modeling in the global carbon simulations. One major issue is that, although the importance of photoperiod has been addressed by many papers, such as K\"{o}rner and Basler 2010 science paper, as well as papers the authors cited, how the photoperiod plays its role during the phenology processes is still largely unclear. For example, the interactive effect among photoperiod, chilling and forcing in the spring leaf-out processes, and the relative importance among these drivers for phenological processes. In one recently study, the authors proposed a framework that the photoperiod interacts with chilling and forcing (GDD) to modify the leaf-out dates to ensure the tree leaf out at the right time, and further suggested that the photoperiod plays its role depending on the phenology dates, i.e. whether or not the phenological dates filling into its optimal periods, see Fu et al, gcb (daylength helps the ..).based on such framework, the species-specific difference, as well as spatial difference in photoperiod effect could be explained. Anyway, such integrative effect need to be reviewed clearly.}

\par We thank the reviewer for noting that our review is interesting, and for suggesting the need to more clearly review the effects of photoperiod are interactive with temperature for many species. We have restructured the introduction and now have a paragraph in which we discuss the interactive effects of photoperiod and temperature (Lines 25-35):
\par ``For many organisms, the timing of spring events-- i.e., phenology, including  flowering, bird arrival, egg hatch0ing and myriad other biological activities| is thought to be determined by photoperiod interactively with temperature (Fu et al., 2019; Winkler et al., 2014, see also Box 1).  The strong role of temperature is apparent in recent advances in spring phenology, some of the most widely documented signals of climate change. At a given location on Earth, annual patterns in photoperiod have not changed as climates have warmed. Yet, across taxa, from plants and insects to mollusks and mammals, spring phenology is occurring earlier as temperatures warm, with average shifts of 1.2 to 5.1 days earlier per decade (Bradley et al., 1999; Parmesan and  Yohe,  2003;  Poloczanska  et  al.,  2013;  Root  et  al.,  2003)  or  1.3  to  5.6  days  earlier  per�C  of  warming (Polgar et al., 2013; Wolkovich et al., 2012).  These changes are some of the largest climate change-induced shifts observed, with early spring phenology shifting more rapidly than later season phenology in most cases (Bradley et al., 1999; Menzel et al., 2006).  

\par We have also expanded Box 1 and now discuss the interactive effect of photoperiod and temperature in detail in Lines 306-321:
``Growth chamber experiments highlight that responses to photoperiod vary depending on temperature conditions.  For example, more rapid advancement of budburst was observed under long versus short days with low chilling, than with high chilling in \emh{Betula payrifera} (Hawkins and Dhar, 2012) (Fig.  4).  Similarly, across species, as chilling accumulates from winter to spring, sensitivity to forcing and photoperiod sensitivity can decrease (Malyshev et al., 2018).  Frequently, long photoperiods can compensate for low amounts of chilling, resulting in enhanced cell growth (Heide, 1993a; Myking and Heide, 1995; Caffarra et al., 2011b). Woody plant growth chamber experiments also demonstrate that, though photoperiod responses are common,316they are variable (Fig.  4).  Responses to photoperiod differ by species (e.g., Basler and K\"{o}rner, 2012, 2014; Flynn and Wolkovich, 2018; Heide, 1993b; Howe et al., 1996; Zohner et al., 2016).  For example, with longer318chilling treatments some species seem insensitive to daylength (e.g., \emph{Hammamelis} spp., \emph{Prunus} spp., Zohner et  al.,  2016),  whereas  others  (e.g., \emph{Fagus} spp.,  Fig.   5A)  seem  to  be  highly  sensitive  to  daylength,  even with  long  chilling  treatments  (Zohner  et  al.,  2016).   In  addition,  some  species  demonstrate  a  response  to photoperiod opposite to that typically observed: \emph{Tilia},  for example,  showed delayed budburst with longer daylengths (Fig.  4, Ashby et al., 1962).  Photoperiod sensitivity also varies by population and ecotype (e.g., Partanen et al., 2005) (Fig.  4).  For example, photoperiod effects on budburst were more significant for lower latitude populations of \emph{Betula pendula} and \emph{B. pubescens} (Partanen et al., 2005)."

\par We have added some additional references to this box in the main text; and we have added the following text to more clearly review the integrative nature of phenology, including adding relevant references such as the one the reviewer mentions, as well as papers by Malyshev et al (2018) and Lundell et al (2020):
\begin{itemize}
\item Lines 25-27: ``For many organisms, the timing of spring events-- i.e., phenology, including flowering, bird arrival, egg hatching and myriad other biological activities-- is thought to be determined by photoperiod interactively with temperature (Fu et al., 2019; Winkler et al., 2014; see also Box 1)." 
\item Lines 125-131: "Daylength, often in combination with temperature, can play a role in controlling critical biological functions, including vegetative growth, cell elongation, budburst, and flowering in plants (Fu et al 2019; Heide and Snsteby, 2012; Heide, 2011; Hsu et al., 2011; Sidaway-Lee et al., 2010; Mimura and Aitken, 2007; Linkosalo and Lechowicz, 2006; Erwin, 1998; Ashby et al., 1962) and growth rate, maturation, reproduction, migration, and diapause in animals (Dunn, 2019; Winkler et al 2014; Zydlewski et al., 2014; Dardente, 2012; Tobin et al., 2008; Bradshaw and Holzapfel, 2006; Ben-David, 1997; Muir et al., 1994; Saunders and Henderson, 1970; Dawbin, 1966).
\item Lines 134-136: ''The direction and magnitude of responses will vary, however, because of variation in photoperiod sensitivity, and because photoperiod often interacts with other environmental drivers, such as temperature, to affect phenology (Box 1)."
\end{itemize}

\par \emph{In addition, the authors suggested that the experimental results could be coupled into the modelling approaches, however the phenological experimental studies might be underestimated, comparing to its natural response, see Wolkovich et al, 2012 nature, how to improve the experimental study to be more reliable and reality is thus important, rather only suggesting couples the experimental results into the modeling approaches. In the review paper in H\"{a}nninen et al, 2019, Trends in plant science, in which the authors suggested even the processed-based modeling is still unreliable, because the key sub-processes are omitted during the experiments and thus physiological experiments are needed, for example controlling the dates of end endodormancy (Chine et al, 2016 gcb).}
\par \emph{H\"{a}nninen H, Kramer K, Tanino K, et al. Experiments are necessary in process-based tree phenology modelling[J]. Trends in plant science, 2019, 24(3): 199-209.}
\par \emph{Fu, Y H. et al. Daylength helps temperate deciduous trees to leaf-out at the optimal time[J]. Global change biology, 2019, 25(7), 2410-2418.}
\par \emph{Chuine, I. et al. Can phenological models predict tree phenology accurately in the future? the unrevealed hurdle of endodormancy break. Glob. Change Biol. 2016, 22, 3444-3460} % EMW: I think we could do more here! See my edits to Box 1.
\\
\par We thank the reviewer for pointing out some important challenges and needs in phenology research! We have added text to our manuscript suggesting the need for additional experimental research, in addition to incorporating these results into modelling approaches. We have also added some text to emphasize some of the physiological research needed, as the reviewer suggests, and we have added citations to the suggested references. Below we summarize the additions we have made:
\begin{itemize}
\item Lines 25-27: We have added the Fu et al 2019 reference: ``For many organisms, the timing of spring events|i.e., phenology, including  flowering, bird arrival, egg hatching and myriad other biological activities| is thought to be determined by photoperiod interactively with temperature (Fu et al., 2019; Winkler et al., 2014, see also Box 1)."
\item Lines 37-38 We have added the Fu et al 2019 reference: ``With additional climate change, photoperiod will limit phenological shifts of certain
38 species such that they will not track rising temperatures (Fu et al.,2019.."
\item Lines 69-71 We have added the Fu et al 2019 reference: ``In addition, the role of photoperiod is less understood in spring phenology compared with autumn phenophases, but recent studies showing declines in responses of spring budburst to warming (e.g. Fu et al., 2019;.."
\item Lines 125-127: We have added the Fu et al 2019 reference: ``Daylength, often in combination with temperature, can play a role in controlling critical biological functions,including vegetative growth,  cell elongation,  budburst,  and  flowering in plants (Fu et al., 2019;..."
\item Lines 137-139: We have added the Fu et al 2019 reference: ``The climate change-induced trend toward ever earlier springs means that experienced photoperiod may increasingly approach threshold photoperiod for many species, constraining their ability to respond to additional warming (Fu et al., 2019.."
\item Lines 225-227: We have added the following text citing the Chuine et al 2016 and H\"{a}nninen et al 2019 references: ``For many species, additional experimental physiological research is necessary,  since the dormancy-breaking processes that photoperiod affects often require microscopy and240detailed physiological approaches to observe (Hanninen et al., 2019; Chuine et al 2016)."
\end{itemize}
\\
\par \emph{Line 7-9, Larger spatial photoperiod shifts than temporal shifts was reported... does it suggest the local climate is more important than air temperature? or the interactive effect among environmental cues plays more important role than air temperature only?}
\\
\par We thank the reviewer for pointing out that this wording may have been unclear. We do not believe that our findings can be used to infer anything about the relative importance of local climate versus air temperature. Our finding that temporal shifts result in larger changes in experienced photoperiod are a result of the way that photoperiod varies with latitude and seasonally, from winter to spring (as shown in Figure 1). We have reworded this sentence slightly to try to make this more clear. The sentence (Lines 7-9) now reads:
``We synthesize published studies to show that impacts on experienced photoperiod from temporal shifts could be orders of magnitude larger than from spatial shifts (e.g., 1.6 hours of change for expected temporal versus only one minute for latitudinal shifts)."
\\
\par \emph{L152-153, yes, but the rate of frost damage may also increase...}
\par The reviewer, who refers to our statement (now Lines 159) that ``For example, a species or population that is relatively insensitive to photoperiod can take
advantage of warmer springs by having an earlier start to its growing season," makes an excellent point. 
We now mention this in Lines XXX, where we write: ``For  example,  relying  on  a  threshold  photoperiod  (see Glossary), rather than temperature alone, may prevent woody plants from leafing out during `false spring' events and experiencing  frost  damage  (unusually  warm  periods  during  winter  that  are  followed  by  a  return  of  cold temperatures (Gu et al 2008)." HELP: If you have ideas, where to work something like this in, please tell me! %EMW: See my edits ... 
\\
\par \emph{L158-159, also identifying the species-specific photoperiod sensitivity is important...}
\par The reviewer makes an excellent point! We have incorporated this idea, so the statement now reads (Lines 165-167):
``To identify where, when, and how communities may be altered, quantifying species-specific photoperiod sensitivity and developing methods for incorporating photoperiod into forecasting future phenology are critical."

******************************************
\par \emph{\bold{Editor's comments to the author(s):}}

\par \emph{Please consider the serious concerns of the referees, especially those of referee 1}
\\
\par We thank the editor for considering this manuscript and have worked to address all concerns, as we detail in our point-by-point responses above.
\newline

\underline{\bold{References included in responses to reviewers}}
Allona I, Ramos A, Ib��ez C, Contreras A, Casado R, Aragoncillo C. Molecular control of winter dormancy establishment in trees: a review. 2008. Spanish Journal of Agricultural Research. 6(S1):201-10.

Azeez A, Sane AP. Photoperiodic growth control in perennial trees. Plant signaling & behavior. 2015 ;10(12):e1087631.

Chamberlain, C.J., et al. 2019. Rethinking false spring risk.  \emph{Global Change Biology}.
K\"{o}rner, C, Basler D. Warming, photoperiods, and tree phenology response. Science.  2010b.  329:278{278.
Chuine I; Morin X, Bugmann H. 2010, Warming, Photoperiods, and Tree Phenology. Science. 329:278.
Fu, Y.H., et al. 2019. Daylength helps temperate deciduous trees to leaf-out at the optimal time. \emph{Global change biology}.

Fu YH, Zhao H, Piao S, Peaucelle M, Peng S, Zhou G, Ciais P, Huang M, Menzel A, Pe�uelas J, Song Y. Declining global warming effects on the phenology of spring leaf unfolding. Nature. 2015 Oct;526(7571):104-7.

Fudickar AM, Greives TJ, Atwell JW, Stricker CA, Ketterson ED. Reproductive allochrony in seasonally sympatric populations maintained by differential response to photoperiod: implications for population divergence and response to climate change. The American Naturalist. 2016 Apr 1;187(4):436-46.

G\"{u}sewell S, Furrer R, Gehrig R, Pietragalla B. Changes in temperature sensitivity of spring phenology with recent climate warming in Switzerland are related to shifts of the preseason. Global change biology. 2017 Dec;23(12):5189-202.

K\"{o}rner, C, Basler D. Phenology under global warming. Science. 2010a. 327(5972):1461-2.

K\"{o}rner, C, Basler D. Warming, photoperiods, and tree phenology response. Science.  2010b.  329:278.

Lagercrantz UL. At the end of the day: a common molecular mechanism for photoperiod responses in plants?. Journal of experimental botany. 2009 Jul 1;60(9):2501-15.

Lundell R, H�nninen H, Saarinen T, �str�m H, Zhang R. 2020. Beyond rest and quiescence (endodormancy and ecodormancy): A novel model for quantifying plant?environment interaction in bud dormancy release. Plant, Cell \& Environment. 43(1):40-54.

Richardson, A.D., et al. 2018. Ecosystem warming extends vegetation activity but heightens vulnerability to cold temperatures. \emph{Nature}, 560: 368.

Yu H, Luedeling E, Xu J. Winter and spring warming result in delayed spring phenology on the Tibetan Plateau. Proceedings of the National Academy of Sciences. 2010 Dec 21;107(51):22151-6.

\end{document}

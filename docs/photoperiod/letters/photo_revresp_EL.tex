\documentclass[11pt,a4paper]{letter}
\usepackage[top=1.00in, bottom=1.0in, left=.75in, right=0.75in]{geometry}
\usepackage{graphicx}
\usepackage{natbib}
\usepackage{gensymb}
\address{1300 Centre Street \\ Boston, MA, 20131}
\begin{document}
\bibliographystyle{/Users/aileneettinger/citations/Bibtex/styles/nature.bst}

\begin{letter}{}
\includegraphics[width=0.5\textwidth]{/Users/aileneettinger/Dropbox/Documents/Work/AA_heading.pdf}
\pagenumbering{gobble}



\opening{Dear Editors:}
Please consider our paper, entitled  ``Spatial and temporal shifts in photoperiod with climate change'' as a Review \& Synthesis in \emph{Ecology Letters}. This manuscript is a revised version of manuscript  ELE-01441-2019. We have incorporated the suggestions of the referees and editor, as detailed in the enclosed point-by-point response. 

As you may remember, our paper addresses an urgent ecological question: What are the implications of the altered photoperiod that plants and animals experience as they shift their ranges and seasonal activities with climate change? The two most-observed biological impacts of climate change are shifts in space (range shifts) and time (phenological shifts). Both alter experienced photoperiod, which could dramatically affect performance and fitness. However, the magnitude of effects from shifts in photoperiod with climate change are unknown or unquantified for the vast majority of species.  We address this need by synthesizing the large body of controlled climate experiments that test effects of temperature and photoperiod on spring phenology in plants \emph{(1)}. Our review differs from previous work \emph{(e.g., 2)} in that it quantifies expected changes in experienced photoperiod due to shifts in space versus time. Though we include a detailed discussion of woody plants, our review is broadly relevant, as photoperiod acts as a cue for the spring emergence and migration timing of diverse species, and altered photoperiod can affect development, growth, and fitness for plants, insects, fish, and mammals, among other organisms. 

All three reviewers felt our review addressed an interesting and important topic. Beyond that, there were some differences in opinion. Reviewer 2 felt our manuscript is a good review of what is known about the effects of photoperiod in the context of climate change, with clear ideas, good discussion, relevant literature, and useful figures. Reviewer 1, on the other hand, felt that our manuscript was difficult to follow, in part because the figures were poorly explained, and that it should focus on autumn rather than spring phenology. The reviewers also made some suggestions for additional topics that should be better integrated into the manuscript. We have addressed by adding text to the manuscript. Reviewer 2 noted that the previous version of the manuscript was mostly plant (and tree) oriented, and suggested some areas in the text where additional emphasis could be placed on the relevence to animals. We have following the suggestions of this reviewer, and added several new animal references about effects of photoperiod on reproduction. Reviewer 3 felt the need for improvements to experiments should be better addressed, and specifically that the need for additional plant physiological experiments should be emphasized. To address these concerns, we have added text, modified the main text and figure legends, and added references. 




 Please see our detailed, point-by-point response below for more information on the changes. 
.


Sincerely,\\

\includegraphics[scale=.4]{/Users/aileneettinger/Dropbox/Documents/Work/AileneEttingerSignature.png} \\
Ailene Ettinger
\begin{footnotesize}\\
Quantitative Ecologist, The Nature Conservancy- Washington Field Office\\
Visiting Fellow, Arnold Arboretum of Harvard University 
\end{footnotesize}
\\
\\
\noindent \emph{References in cover letter}

\begin{footnotesize}
\begin{enumerate}
\item Wolkovich, E.M.,  et al. 2019. Observed Spring Phenology Responses in Experimental Environments (OSPREE). Knowledge Network for Biocomplexity. urn:uuid:b2ab2746-b830-4
\item Saikkonen, K., et al. 2012. Climate change-driven species' range shifts filtered by photoperiodism. \emph{Nature Climate Change}, 2:239.
\item Chamberlain, C.J., et al. 2019. Rethinking false spring risk.  \emph{Global Change Biology}.
\item Richardson, A.D., et al. 2018. Ecosystem warming extends vegetation activity but heightens vulnerability to cold temperatures. \emph{Nature}, 560: 368.
\item Fu, Y.H., et al. 2019. Daylength helps temperate deciduous trees to leaf-out at the optimal time. \emph{Global change biology}.
\item Flynn D.F. \& Wolkovich EM. 2018. Temperature and photoperiod drive spring phenology across all species in a temperate forest community. \emph{New Phytologist}, 219:1353-62.
\item Wolkovich, E.M.,  et al. 2011. Warming experiments underpredict plant phenological responses to climate change. \emph{Nature} 485: 494.
\item Ettinger, A. K., et al. 2019. "How do climate change experiments alter plot?scale climate?. \emph{Ecology letters}, 22: 748-763.
 
\end{letter}


\title{Response to Reviewers}
\emph{Reviewer Comments are in italics.} Author responses are in plain text.\\

 \emph{\bold{Reviewer \#1 (Comments for the Authors)}}\\

\emph{I am happy to review this manuscript as a Reviews and Synthesis paper to Ecology Letters. Basically, this paper aims at answering the three questions: 1. How will climate change alter the photoperiod experienced by organisms? 2. What are the implications of altered photoperiods for biological responses to climate change? 3. Can research apply data from experiments that alter photoperiod to aid in forecasting biological implications of climate change? Absolutely, these three questions are very important for understanding the changes of photoperiods and its impacts to biological phenophases. However, it is pity that I found very limited new knowledge by reading this paper, and many places are very difficulty to follow. I have several major concerns as followings. }

\par We thank the reviewer for taking the time to review our paper and provide detailed comments. We appreciate that the reviewer agrees the questions addressed in our paper our very important. We have addressed the reviewer's concerns throughout the manuscript. In particular, we have worked to more clearly articulate the new knowledge and perspectives of our paper and have sought to make it easier to follow.


\par \emph{1. The authors seem to focus on the responses of spring phenophases to photoperiods, but it may be a wrong direction. There is obvious asymmetry of radiation and temperature on seasonal changes. The peak of radiation is always earlier than that of temperature, and which means the limited environmental variable in Spring is temperature not radiation. Therefore, the photoperiod is not important for plant phenology in Spring. On contrary, in autumn, the radiation or daylength is very low when the temperature keeps still very high. However, I did not find some reviews on the changes in autumn phenophases in response to photoperiod.}


\par The reviewer is correct that our work focuses on spring, not autumn, phenology; we agree that a review on autumn plant phenology responses would also be interesting and useful. We chose to focus on the intersection of photoperiod with spring phenology in part because it has been less studied compared to autumn phenophases. In addition, this topic has received growing interest (e.g., Chamberlain et al 2019, Fu et al 2019, Richardson et a 2018), given recent studies hiilighting declining responses of spring budburst to warming (e.g., Fu et al 2019, G�sewell et al 2017, Yu et al 2010). One possible cause of these declines in sensitivity could be photoperiod thresholds. To date, the role of photoperiod has received far more detailed attention for end-of-season activities, such as growth cessation in the fall, than for spring activities. Though photoperiod cues dominate in the fall for many organisms, as the reviewer suggests, fall phenology responses to climate change have been muted. In contrast, spring phenology responds strongly to temperature and thus has advanced substantially with warming---causing cascading, and generally unexplored, effects on photoperiod experienced at the start of spring. Thus, we wish to shine a light on the potential implications to spring phenology of altered photoperiod cues with climate change. We also suggest that there are many opportunities for additional research in this field, given the wealth of data from growth chamber experiments on woody plant spring phenology. 


\par \emph{2. the abstract is very week because it fails to summarize the main content of this review manuscript. The authors are going to review the study progresses on photoperiods and its impacts, therefore, the abstract should also focus on the three key scientific questions. However, the current abstract does not highlight the key results and conclusions but repeat the background. }
We have modified the abstract....

\par \emph{3. The introduction starts from the impacts of climate warming on phenophases (the first paragraph). I would like to read the importance of photoperiods first, and then to know the connections between temperature and photoperiods. }
\par We thank the reviewer for this suggestion. We have re-written the beginning of the introduction as the reviewer suggests...

\par \emph{4. Line 36: these two references did not support your statement "Some studies suggest that, with additional warming, photoperiod will limit phenological shifts of certain species such that they will not track rising temperatures (e.g., by leaf out earlier in the spring, Korner and Basler, 2010; Way and Montgomery, 2015)." These two papers are review papers, and both describe the contrary conclusion with the authors. For example, the first paper states "However, no study has demonstrated that photoperiod is more dominant than temperature when predicting leaf senescence (1), leafing, or flowering" The authors should double check and be cautious for all conclusions in the manuscript. Therefore, I doubt the main topic in this review paper as I mentioned before that the impacts of photoperiods is very limited to phenophases in Spring.} 


\par The reviewer has made statements in this comment that we feel misrepresents our work. The reviewer incorrectly attributes a quote to Korner and Basler (2010b)  (``However, no study has demonstrated that photoperiod is more dominant than temperature when predicting leaf senescence (1)." This quote is from a "Response" to Korner and Basler (2010a), written by Isabelle Chuine, Xavier Morin, and Harald Bugmann (Chuine et al 2010). Chuine et al (2010) make important points in their response to Korner and Basler (2010a) but we did not reference their writing in our manuscript. The reviewer seems to suggest that we have misquoted the two references we cite (Korner and Basler, 2010b; Way and Montgomery, 2015). We disagree. These two references are examples of writing that describe the idea that, with future warming, photoperiod cues may limit plant responses to temperature. Both writings we cite do this; they do not describe conclusions that are contrary to ours, as the reviewer states. Please see below for direct quotes from these two citations. We encourage the reviewer and editor to consult the references directly to confirm the accuracy of our summary of the authors' statements:
\begin{itemize}
\item From Korner and Basler (2010b): ``Warm temperatures are most effective in promoting budburst only after the photoperiod requirement of adequately chilled buds is met (3, 6). This
process prevents trees from sprouting before the risk of freezing damage is over. Photoperiodism in trees has been known for almost 100 years (7, 8), but is an often ignored environmental cue when predicting the effects of a future climate."
\item From Way and Montgomery (2015): ``Here, we discuss how day length may limit the ability of tree species to respond to climate warming in situ, focusing on the implications
of photoperiodic sensing for extending the growing season and affecting plant phenology and growth, as well as the potential role of photoperiod in controlling carbon uptake and water fluxes in forests."
\item From Way and Montgomery (2015): ``Organisms that rely on photoperiodic cues for sensing the arrival of spring and the approach of winter may have constrained responses to warming, which would limit expected extensions of the growing season in mid- and high latitude forests (Saikkonen et al. 2012).
\item From Way and Montgomery (2015): ``There is also evidence for direct photoperiodic sensitivity of bud burst in some species (e.g. F. sylvatica; Heide 1993a): at shorter photoperiods, it takes longer for buds to burst than at long photoperiods, indicating that long photoperiods enhance dormancy release (Heide 1993a,b). Tree species that fall into this category are likely to be strongly limited in their ability to respond to warmer springs by breaking bud earlier."
\item From Way and Montgomery (2015): ``A reliance on photoperiod as a seasonal cue for growth and physiological activity may either constrain forest productivity,
via limitations on photosynthetic capacity, migration potential or growing season length, or permit longer growing seasons and greater tree growth in individuals that migrate to higher latitudes."
\end{itemize}

\par We do agree with the reviewer that our characterization of the papers as "studies" may have been misleading, as the referenced articles are not experiments. One is  review paper summarizing many observational and experimental studies (Way and Montogmery 2015). The other (Basler and Korner 2010b) is a response to a response to a perspective (Basler and Korner 2010a). It was not our intention to be misleading. To address this, we have removed the original Korner and Basler citation (Korner and Basler 2010b) , and added instead a long-term observational study (Fu et al 2015), a growth chamber experiment (Basler \& K�rner 2012), and a perspective (K�rner, C. & Basler 2010a), in addition to the review paper we originally cited (Way and Montgomery 2015). We have also modified the text slightly. The new phrase says:
``It has been suggested that, with additional climate change, photoperiod will limit phenological shifts of certain species such that they will not track rising temperatures (Fu et al 2015, Way and Montgomery 2015, Basler \& K�rner 2012, K�rner \& Basler 2010)."

\par \emph{5. Figure 1 and 2 are two the most important results in this review paper. However, the authors fail to introduce their importance clearly. Figure 1 only shows the temporal changes of daylength at two locations, and I have no idea about other information in the figure. Figure 2 is very low level greenness map, and I did not find any useful information at all. In addition, these two figures can not support the conclusion in the text very well. For example, how can the authors arrive at this conclusion "A general pattern of longer photoperiod at green-up toward the poles is consistent across years (Fig. 2b) and green-up does not appear to occur at daylengths less than 10 hours."? (Line 82-83)}
\par We thank the reviewer for pointing out that our explanations of the figures were unclear in the previous version. In this new version, we have adjusted the Figure legends to improve clarity. Figure 2 legend now says....


\par \emph{6. the authors fail to summarize the knowledge on "How will climate change alter the photoperiod experienced by organisms?" I only get very limited knowledge on current conditions of photoperiods experiences by organisms and future changes. In addition, these information focus on the local scale, no global pattern shown. I do not think this kind review can promote science advances. Same feelings for other two sections.} 

\par \emph{7. Figure 3 and 5 are results derived from OSPREE database, but it is very hard to understand how the experiments have been conducted and if the conclusions are reliable because the authors did not introduce them adequately.} 

\par \emph{8. There are too many conclusions no evidence supported to list them all.}


Chamberlain, C.J., et al. 2019. Rethinking false spring risk.  \emph{Global Change Biology}.
Fu, Y.H., et al. 2019. Daylength helps temperate deciduous trees to leaf-out at the optimal time. \emph{Global change biology}.
Fu YH, Zhao H, Piao S, Peaucelle M, Peng S, Zhou G, Ciais P, Huang M, Menzel A, Pe�uelas J, Song Y. Declining global warming effects on the phenology of spring leaf unfolding. Nature. 2015 Oct;526(7571):104-7.
G�sewell S, Furrer R, Gehrig R, Pietragalla B. Changes in temperature sensitivity of spring phenology with recent climate warming in Switzerland are related to shifts of the preseason. Global change biology. 2017 Dec;23(12):5189-202.

K�rner C, Basler D. Phenology under global warming. Science. 2010a. 327(5972):1461-2.
K�rner C, Basler D. Warming, photoperiods, and tree phenology response. Science.  2010b.  329:278{278.

Richardson, A.D., et al. 2018. Ecosystem warming extends vegetation activity but heightens vulnerability to cold temperatures. \emph{Nature}, 560: 368.

Yu H, Luedeling E, Xu J. Winter and spring warming result in delayed spring phenology on the Tibetan Plateau. Proceedings of the National Academy of Sciences. 2010 Dec 21;107(51):22151-6.

\par \emph{\bold{Reviewer \#2 (Comments for the Authors)}}
\par \emph{Manuscript is a good review of what is known about the effects of photoperiod in the context of climate change. Although the title doesn't make this clear, it's mostly plant (and tree) oriented.  I think I made a few notes about places where relevance to animals could be included.  It's pretty well written, although I've flagged some editorial issues on the PDF.  I think the relevant literature is cited, and the figures are useful and clear.  The ideas are presented clearly and the discussion is good.  See a few other comments on the PDF. } 
\par We thank the reviewer for these comments. We have incorporated the relevance to animals in places where the reviewer suggested  (e.g., by mentioning the affect of photoperiod on reproduction in animals, and by adding additional references focused on photoperiod effects on animals). We feel this has strengthened the manuscript by demonstrating the broad range of taxa affected by photoperiod. We have also addressed the editorial issues noted in the pdf, which are summarized below:

\par Line  \emph{Do  you mean latitudinal shifts?}
Yes! We have replaced ``spatial'' with ``latitudinal'' to be more specific. The phrase now says ``(e.g., 1.6 hours of change for expected temporal shifts versus only one minute for latitudinal shifts)."
\par Line 21: We replaced "climate change induced" with "climate change-induced" as suggested by the reviewer.
\par Line 41: \emph{cues? The cues won't change. maybe 'responses'?}
\par We thank the reviewer for pointing out a lack of clarity here. We have rewritten the phrase, as suggested, which now reads: ``The extent to which daylength constrains phenology will depend in part on how rapidly photoperiod responses can acclimate or adapt to new environmental conditions, which remains poorly understood (Grevstad and Coop, 2015)."}
\par Line 52-53 \emph{Do you mean this to include reproduction? If not, add reproduction to that list. }
\par We have added reproduction to the list, which now reads: ``...since daylength can affect the timing of development (Grevstad and Coop, 2015; Muir et al., 1994), migration
(Dawbin, 1966), reproduction (Dunn, 2019; Dardente, 2012; Ben-David, 1997), and other important responses. "
\par Line 62: We have replaced ``research" with ``researchers" as suggested by the reviewer.
\par Line 114: We have replaced ``if" with ``whether" as suggested by the reviewer.
\par Line 121-125: We have added reproduction to the list of animal responses affected by daylength, as suggested by the reviewer. This section now says:
``Daylength can play a role in controlling critical biological functions, including vegetative growth, cell elongation, budburst, and flowering in plants (Heide and Sonsteby, 2012; Heide, 2011; Hsu et al., 2011; Sidaway-Lee122et al., 2010; Mimura and Aitken, 2007; Linkosalo and Lechowicz, 2006; Erwin, 1998; Ashby et al., 1962) and growth rate, maturation, reproduction, migration, and diapause in animals (Dunn, 2019; Zydlewski et al.,2014; Dardente, 2012; Tobin et al., 2008; Bradshaw and Holzapfel, 2006; Ben-David, 1997; Muir et al., 1994; Saunders and Henderson, 1970; Dawbin, 1966)."
\par Lines 156-157: We have added hyphens, as suggested, so that the phrase now reads: ``slower-growing or later-emerging ones"
Referee: 3
\par Line 168: We have added a semi-colon, as suggested.
\par Line 182: We have replaced ``if" with ``whether", as suggested.
\par Line 199: We have added a hyphen, as suggested, changing ``fine scale" to ``fine-scale". 
\par Lines 233-257 (Glossary): We have ensured that all definitions end with periods, to be consistent, as suggested by the reviewer. 
\par Line 290: We have made the suggested change to our wording. The phrase now reads: ``..some species demonstrated a response to photoperiod opposite to that typically observed..."

\par Line 333: We have added the published of this book to the reference, as suggested by the reviewer. Thank you for noticing this error!
\par Line 359: We have capitalized the genus name, and italicized the genus, species, and variety names. Thank you for noticing this error!
\par Line 407: We have italicized the genus and species names. Thank you for noticing this error!

\par Figure 1: We have replaced ``Spatial shift" with ``Latitudinal shift" in the figure legend, as suggested.


\par \emph{\bold{Reviewer \#3 (Comments for the Authors)}}

\par \emph{The shifts of photoperiod is not only the result of phenology changes, such as earlier in spring leaf-out and geographic shifts in species distribution, but also the reason/driver of phenology change, for example the shortening photoperiod associating with earlier leaf-out reduced the temperature sensitivity. I do think this is an interesting review in the photoperiod shifts, although many papers have discussed such photoperiod effect on phenology, and agree that the photoperiod effect should be coupled into the phenology modeling, especially considering the larger uncertainty of phenology modeling in the global carbon simulations. One major issue is that, although the importance of photoperiod has been addressed by many papers, such as Korner and Basler 2010 science paper, as well as papers the authors cited, how the photoperiod plays its role during the phenology processes is still largely unclear. For example, the interactive effect among photoperiod, chilling and forcing in the spring leaf-out processes, and the relative importance among these drivers for phenological processes. In one recently study, the authors proposed a framework that the photoperiod interacts with chilling and forcing (GDD) to modify the leaf-out dates to ensure the tree leaf out at the right time, and further suggested that the photoperiod plays its role depending on the phenology dates, i.e. whether or not the phenological dates filling into its optimal periods, see Fu et al, gcb (daylength helps the ..).based on such framework, the species-specific difference, as well as spatial difference in photoperiod effect could be explained. Anyway, such integrative effect need to be reviewed clearly.}

\par We thank the reviewer for noting that our review is interesting, and for suggesting the need to more clearly review that effect of photoperiod is interactive with temperature for many species. We have added the following text to more clearly review the integrative nature of phenology:
-LINES XX

\par \emph{In addition, the authors suggested that the experimental results could be coupled into the modelling approaches, however the phenological experimental studies might be underestimated, comparing to its natural response, see Wolkovich et al, 2012 nature, how to improve the experimental study to be more reliable and reality is thus important, rather only suggesting couples the experimental results into the modeling approaches. In the review paper in H�nninen et al, 2019, Trends in plant science, in which the authors suggested even the processed-based modeling is still unreliable, because the key sub-processes are omitted during the experiments and thus physiological experiments are needed, for example controlling the dates of end endodormancy (Chine et al, 2016 gcb).}
\par \emph{H�nninen H, Kramer K, Tanino K, et al. Experiments are necessary in process-based tree phenology modelling[J]. Trends in plant science, 2019, 24(3): 199-209.}
\par \emph{Fu, Y H. et al. Daylength helps temperate deciduous trees to leaf-out at the optimal time[J]. Global change biology, 2019, 25(7), 2410-2418.}
\par \emph{Chuine, I. et al. Can phenological models predict tree phenology accurately in the future? the unrevealed hurdle of endodormancy break. Glob. Change Biol. 2016, 22, 3444-3460}

\par We thank the reviewer for pointing out some important challenges and needs in phenology research! We have added text to our manuscript suggesting some needs in experimental research, in addition to incorpoating these results into modelling approaches. We have also added some text to emphasize some of the physiological research needed, as the reviewer suggests. Below we summarize the additions we have made:


\par \emph{Line 7-9, Larger spatial photoperiod shifts than temporal shifts was reported... does it suggest the local climate is more important than air temperature? or the interactive effect among environmental cues plays more important role than air temperature only?}
\par \emph{L152-153, yes, but the rate of frost damage may also increase...}
\par \emph{L158-159, also identifying the species-specific photoperiod sensitivity is important...}

******************************************
Editor's comments to the author(s):

Editor
Editors Comments for the Author(s):
Please consider the serious concerns of the referees, especially those of referee 1
\par We thank the editor for considering this manuscript.

\end{document}

\documentclass[11pt,a4paper]{letter}
\usepackage[top=1.00in, bottom=1.0in, left=0.75in, right=0.75in]{geometry}
\usepackage{graphicx}
\usepackage{natbib}

\begin{document}

\begin{letter}{}
\includegraphics[width=0.2\textwidth]{/Users/aileneettinger/Dropbox/Documents/Work/AA_heading.pdf}

\opening{Dear Dr. Chase and Dr. Hillebrand:}
We are submitting a Review \& Synthesis to address an urgent ecological question: What are the implications of the altered photoperiod that plants and animals experience as they shift their ranges and seasonal activities with climate change? The two most-observed biological impacts of climate change are shifts in space (range shifts) and time (phenological shifts). Both alter experienced photoperiod, which could dramatically affect performance and fitness. However, the magnitude of effects from shifts in photoperiod with climate change are unknown or unquantified for the vast majority of species.  
\par We address this need by synthesizing the large body of controlled climate experiments that test effects of temperature and photoperiod on spring phenology \emph{(1)}. Our review differs from previous work \emph{(e.g., 2)} in that it quantifies expected changes in experienced photoperiod due to shifts in space versus time. It is broadly relevant, as photoperiod acts as a cue for the spring emergence and migration timing of diverse species, and altered photoperiod can affect development, growth, and fitness for plants, insects, fish, and mammals, among other organisms. Thus, understanding these changes is critical for scientists studying basic ecology, as well as those who wish to forecast species responses to climate change and use forecasts to develop adaptation strategies. Yet, photoperiod has rarely been included in forecasts of climate change responses, and implications of climate-change-induced shifts in photoperiod are largely unexplored, especially for early-season spring events, where changes will be most dramatic. 
\par This piece is timely and important because of its focus on the intersection of photoperiod and spring phenology (similar to \emph{3,4,5})---this is an area of growing interest given recent studies suggesting that photoperiod may underlie declining responses to warming. To date, the role of photoperiod has received far more detailed attention for end-of-season activities, such as growth cessation in the fall, than for spring activities. Though photoperiod cues dominate in the fall for many organisms, fall phenology responses to climate change have been muted. In contrast, spring phenology responds strongly to temperature and thus has advanced substantially with warming---causing cascading, and generally unexplored, effects on photoperiod experienced at the start of spring. We demonstrate that incorporating photoperiod into forecasts is possible by leveraging existing experimental data: as an example, we show that growth chamber experiments on woody plant spring phenology often have data relevant for climate change impacts (e.g., Fig. 1). 

\par This paper was inspired by a meta-analysis, currently in review at \emph{Nature Climate Change} and attached to this submission per your requirements, that quantifies the effects of chilling, forcing, and photoperiod on budburst phenology of woody plants. Both that meta-analysis and the manuscript submitted here use data from a new database of spring phenology responses \emph{(1)}; the analyses, figures, and writing in the manuscript we submit to you are entirely unique and not under consideration elsewhere. Co-authors are D. Buonaiuto, C. Chamberlain, I. Morales-Castilla, and E. Wolkovich.  Our international team includes researchers well-versed in techniques of controlled climate experiments, having conducted such studies ourselves  \emph{(e.g., 6)}, as well as scientists with expertise in meta-analytical approaches  \emph{(e.g., 7,8)}. We expect our Review \& Synthesis will inspire innovative research to improve mechanistic understanding of photoperiod as a cue for diverse biological processes, as well as a deeper appreciation for the ways that experienced photoperiod may both affect and be affected by species responses to climate change; we hope you will consider it for \emph{Ecology Letters}. We suggest the following potential reviewers: Josep Pe\~nuelas, David Inouye, and Mark Schwartz.

Sincerely,\\

\includegraphics[scale=.4]{/Users/aileneettinger/Dropbox/Documents/Work/AileneEttingerSignature.png} \\
Ailene Ettinger
\begin{footnotesize}\\
Quantitative Ecologist, The Nature Conservancy- Washington Field Office\\
Visiting Fellow, Arnold Arboretum of Harvard University 
\end{footnotesize}
\\
\\
\noindent \emph{References in cover letter}

\begin{footnotesize}
\begin{enumerate}
\item Wolkovich, E.M.,  et al. 2019. Observed Spring Phenology Responses in Experimental Environments (OSPREE). Knowledge Network for Biocomplexity. urn:uuid:b2ab2746-b830-4
\item Saikkonen, K., et al. 2012. Climate change-driven species' range shifts filtered by photoperiodism. \emph{Nature Climate Change}, 2:239.
\item Chamberlain, C.J., et al. 2019. Rethinking false spring risk.  \emph{Global Change Biology}.
\item Richardson, A.D., et al. 2018. Ecosystem warming extends vegetation activity but heightens vulnerability to cold temperatures. \emph{Nature}, 560: 368.
\item Fu, Y.H., et al. 2019. Daylength helps temperate deciduous trees to leaf-out at the optimal time. \emph{Global change biology}.
\item Flynn D.F. \& Wolkovich EM. 2018. Temperature and photoperiod drive spring phenology across all species in a temperate forest community. \emph{New Phytologist}, 219:1353-62.
\item Wolkovich, E.M.,  et al. 2011. Warming experiments underpredict plant phenological responses to climate change. \emph{Nature} 485: 494.
\item Ettinger, A. K., et al. 2019. "How do climate change experiments alter plot?scale climate?. \emph{Ecology letters}, 22: 748-763.
 
\end{letter}
\end{document}

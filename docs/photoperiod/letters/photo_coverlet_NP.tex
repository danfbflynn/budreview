\documentclass[10.5pt,a4paper]{letter}
\usepackage[top=0.75in, bottom=0.75in, left=0.75in, right=0.75in]{geometry}
\usepackage{graphicx}
\usepackage{natbib}

\begin{document}

\begin{letter}{}
\includegraphics[width=0.2\textwidth]{/Users/aileneettinger/Dropbox/Documents/Work/AA_heading.pdf}

\opening{Dear Editor:}
We are submitting a manuscript to \emph{New Phytologist} that addresses an urgent question in plant biology: What are the implications
of the altered photoperiod that plants experience with climate change-induced shifts in their ranges and seasonal activities? The two most-observed biological impacts of climate change are shifts in space (range shifts) and time (phenological shifts). Both alter experienced photoperiod, which could dramatically affect performance and fitness. However, the magnitude of effects from shifts in photoperiod with climate change are unknown or unquantified for most species. This manuscript combines aspects of both a `Research review' and a `Viewpoint,' and thus we believe would be appropriate in either format.
\par We synthesize the large body of controlled environment experiments that test effects of temperature and photoperiod on spring phenology (\emph{1}). Our review differs from previous work  (\emph{e.g., 2,3}) by quantifying expected changes in experienced photoperiod due to shifts in space versus time. Our manuscript is  broadly relevant to \emph{New Phytologist} readers, as photoperiod acts as a cue for the spring emergence of diverse plant species, and altered photoperiod can affect plant development, growth, and fitness. Understanding these changes is critical for scientists studying basic ecology, with extensions to cellular and molecular biology research relating to photoperiod. The work also has relevance to how we forecast plant and ecosystem responses to climate change, since implications of climate change-induced shifts in photoperiod are largely unexplored, especially for early-season spring events, where---we argue---changes may be dramatic.

\par Our paper is timely and important because of its focus on the intersection of photoperiod and spring phenology (similar to \emph{4,5})---this is an area of growing interest given recent studies suggesting that photoperiod may underlie declining responses to warming (\emph{e.g., 6}). To date, the role of photoperiod has received far more detailed attention for end-of-season activities, such as growth cessation in the fall, than for spring activities. Though photoperiod cues dominate in the fall for many organisms, fall phenology responses to climate change have been muted. In contrast, spring phenology responds strongly to temperature and thus has advanced substantially with warming causing cascading, and generally unexplored, effects on photoperiod experienced at the start of spring. We demonstrate that incorporating photoperiod into forecasts is possible by leveraging existing experimental data: as an example, we show that growth chamber experiments on woody plant spring phenology often have data relevant for climate change impacts. 

\par Our international team includes researchers well-versed in techniques of controlled climate experiments, having conducted such studies ourselves (e.g., \emph{7}), as well as scientists with expertise in meta-analytical approaches (e.g., \emph{8,9}). We expect that our `Research review' or `Viewpoint,'will inspire innovative research to
improve mechanistic understanding of photoperiod as a cue for diverse biological processes, as well as a deeper appreciation for the ways that experienced photoperiod may both affect and be affected by plant responses to climate change; we hope you will consider it for \emph{New Phytologist}.

\par Sincerely,\\

\includegraphics[scale=.3]{/Users/aileneettinger/Dropbox/Documents/Work/AileneEttingerSignature.png} \\
Ailene Ettinger
\begin{footnotesize}\\
Quantitative Ecologist, The Nature Conservancy- Washington Field Office\\
Visiting Fellow, Arnold Arboretum of Harvard University 
\end{footnotesize}
 
 \noindent \emph{References in cover letter}

\begin{enumerate}
\item Wolkovich, E.M.,  et al. 2019. Observed Spring Phenology Responses in Experimental Environments (OSPREE). Knowledge Network for Biocomplexity. urn:uuid:b2ab2746-b830-4
\item Saikkonen, K., et al. 2012. Climate change-driven species' range shifts filtered by photoperiodism. \emph{Nature Climate Change}, 2:239.
\item Way DA, Montgomery RA. 2015. Photoperiod constraints on tree phenology, performance and migration in a warming world. \emph{Plant, Cell \& Environment}, 38:1725-36.
\item Richardson, A.D., et al. 2018. Ecosystem warming extends vegetation activity but heightens vulnerability to cold temperatures. \emph{Nature}, 560: 368.
\item Fu, Y.H., et al. 2019. Daylength helps temperate deciduous trees to leaf-out at the optimal time. \emph{Global change biology}.
\item Fu, Y. S. H. et al. 2015. Declining global warming effects on the phenology of spring leaf unfolding. \emph{Nature} 526, 104-107.
\item Flynn D.F. \& Wolkovich EM. 2018. Temperature and photoperiod drive spring phenology across all species in a temperate forest community. \emph{New Phytologist}, 219:1353-62.
\item Wolkovich, E.M.,  et al. 2011. Warming experiments underpredict plant phenological responses to climate change. \emph{Nature} 485: 494.
\item Ettinger, A. K., et al. 2019. How do climate change experiments alter plot-scale climate?. \emph{Ecology letters}, 22: 748-763.
 \end{enumerate}
\end{footnotesize}

\end{letter}
\end{document}

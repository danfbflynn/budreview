% Straight up stealing preamble from Eli Holmes 
%%%%%%%%%%%%%%%%%%%%%%%%%%%%%%%%%%%%%%START PREAMBLE THAT IS THE SAME FOR ALL EXAMPLES
\documentclass{article}

%Required: You must have these
\usepackage{Sweave}
\usepackage{graphicx}
\usepackage{tabularx}
\usepackage{hyperref}
\usepackage{natbib}
\usepackage{pdflscape}
\usepackage{array}
\usepackage{gensymb}
%\usepackage[backend=bibtex]{biblatex}
%Strongly recommended
  %put your figures in one place
%\SweaveOpts{prefix.string=figures/, eps=FALSE} 
%you'll want these for pretty captioning
\usepackage[small]{caption}

\setkeys{Gin}{width=0.8\textwidth}  %make the figs 50 perc textwidth
\setlength{\captionmargin}{30pt}
\setlength{\abovecaptionskip}{10pt}
\setlength{\belowcaptionskip}{10pt}
% manual for caption  http://www.dd.chalmers.se/latex/Docs/PDF/caption.pdf

%Optional: I like to muck with my margins and spacing in ways that LaTeX frowns on
%Here's how to do that
 \topmargin -1.5cm        
 \oddsidemargin -0.04cm   
 \evensidemargin -0.04cm  % same as oddsidemargin but for left-hand pages
 \textwidth 16.59cm
 \textheight 21.94cm 
 %\pagestyle{empty}       % Uncomment if don't want page numbers
 \parskip 7.2pt           % sets spacing between paragraphs
 %\renewcommand{\baselinestretch}{1.5} 	% Uncomment for 1.5 spacing between lines
\parindent 0pt% sets leading space for paragraphs
\usepackage{setspace}
%\doublespacing

%Optional: I like fancy headers
%\usepackage{fancyhdr}
%\pagestyle{fancy}
%\fancyhead[LO]{How do climate change experiments actually change climate}
%\fancyhead[RO]{2016}
 
%%%%%%%%%%%%%%%%%%%%%%%%%%%%%%%%%%%%%%END PREAMBLE THAT IS THE SAME FOR ALL EXAMPLES

%Start of the document
\begin{document}

%\SweaveOpts{concordance=TRUE}
 \bibliographystyle{/Users/aileneettinger/citations/Bibtex/styles/amnat.bst}
\title{Supplemental materials for spatial and temporal shifts in photoperiod with climate change} % perspective paper for OSPREE analyses

\author{A.K. Ettinger, D. Buonaiuto, C. Chamberlain, I. Morales-Castilla, E. Wolkovich}
%\date{\today}%do we need to also add any of the following: D. Flynn, T. Savas, J. Samaha, E. Forrestel? 
\maketitle  %put the fancy title on
%\tableofcontents      %add a table of contents
%\clearpage
%%%%%%%%%%%%%%%%%%%%%%%%%%%%%%%%%%%%%%%%%%%%%%%%%%%

\section*{Supplemental Methods}
\par \textbf{Greenup differences (Figure 2)}
\par Satellite images are combined with algorithms---e.g. MODIS Land Cover Dynamics--- to identify the dates on which phenophases transition from one to the next. Using data from the MODIS sensor (available at: http... DAN, could you please provide the website from where the maps were extracted), we extracted spatial data for North American and Western European green-up---the beginning of seasonal greening---for the years 2009 and 2012. Green-up dates are calculated on the basis of the onset of the Enhanced Vegetation Index \citep{huete2002overview}. From green-up maps for each year we derived the photoperiod corresponding to each pixel (according to its geographic coordinates and day of the year), using R function XXX in package XXX (please Dan, fill these gaps) (see Fig. 2a,b in main text). Finally, we mapped spatial patterns of temporal shifts in green-up comparing an early and late spring years. To do so, we simply substracted the 2013 green-up map to the 2009 one. The spatial resolution corresponding to the maps is of 0.1 x 0.1 degrees.
%% IMC - I'm not sure what level of detail we want to give here. Let me know if you want me to be more explicit/clear.

%%%%%%%%%%%%%%%%%%%%%%%%%%%%%%%%%%%%%%%%
\end{document}
%%%%%%%%%%%%%%%%%%%%%%%%%%%%%%%%%%%%%%%%

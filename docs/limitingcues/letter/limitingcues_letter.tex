\documentclass[12pt,a4paper]{letter}
\usepackage[top=1.00in, bottom=1.0in, left=1in, right=1in]{geometry}
\usepackage{graphicx}

\begin{document}
\begin{letter}{}
\includegraphics[width=0.1\textwidth]{/Users/Lizzie/Documents/Professional/images/letterhead/ubc/UBClogo.jpg}
\pagenumbering{gobble}
\opening{Dear Dr. EsteemedEditor:}
\vspace{1.5ex}\\
Please consider our manuscript, entitled ``Limiting cues: How spring warming, winter chilling and daylength shape plant climate change responses,'' for publication as a Review in \emph{your special journal}. 
\vspace{1.5ex}\\

Controlled environment studies can help predict non-linear responses by allowing researchers to examine the effects of one cue with the others held constant, and examine interactive effects, given the appropriate study design. Such experiments may be especially useful for forecasting if they contain enough variation in treatments to capture precisely where non-linearities occur, and are designed across a range of levels relevant to current versus future conditions \citep{shen2015}. Indeed, one of the major advantages of experiments is that they allow treatments outside of the historical range of a species' or region's climate---an option long-term observational data cannot provide.

We reviewed controlled environment studies over the last seven decades to understand the range of treatments already available, and how they compare to current and future conditions.

cellular staining methods ...
forecasting GCMs

Research on phenology had been conducted for centuries before anthropogenic climate change caused earlier budburst and leafout across much of the globe \citep{Lamb:1948aa,Sparks:1995mv}. Decades of controlled environment studies contributed to our fundamental understanding of the drivers of spring plant phenology. Today, climate change requires leveraging these decades and centuries of research for more accurate predictions that can help humans adapt to warming. \\

We have outlined how researchers could better harness the power of controlled environment experiments to transform our fundamental understanding of phenology and advance forecasting. Controlled environment studies can critically rule out, or support, hypotheses to explain observed discrepancies in long-term data and open up new pathways to use long-term data to understand current trends, helping the field move beyond trying to tease out cues using only long-term data where cues are inherently correlated. While understanding, modeling and predicting interactions among cues and their effects on phenology is challenging, it will yield more accurate predictions---with valuable implications to more realistically assess the effects of climate change on plant biodiversity, including agricultural and forest species. 



\vspace{1.5ex}\\
% Finally, we provide a framework to leverage existing ecological theory to understand how tracking in stationary and non-stationary systems may shape communities, and thus help predict the indirect consequences of climate change.
% % Climate change upends the assumption of stationarity. By causing increases in temperature, larger pulses of precipitation, increased drought, and more storms \citep{ipcc2013}, climate change has fundamentally shifted major attributes of the environment from stationary to non-stationary regimes.
Upon acceptance for publication, data from a systematic literature review included in the paper will be freely available at KNB (knb.ecoinformatics.org); the full dataset is available to reviewers and editors upon request. %This work includes a meta-analysis, so data have been previously published; however, the synthesis of these data and the tables, figures, models, and materials presented in this manuscript have not been previously published nor are they under consideration for publication elsewhere.
\vspace{1.5ex}\\
We recommend the following reviewers: XX.  All authors substantially contributed to this work and approved of this version for submission. The manuscript is approximately XXX words with XX word abstract, X figures, and XX references. It is not under consideration elsewhere. We hope that you will find it suitable for publication in \emph{your special journal} and look forward to hearing from you.
\vspace{1.5ex}\\
Sincerely,\\

\includegraphics[scale=1]{/Users/Lizzie/Documents/Professional/Vitas/Signatures/SignatureLizzieSm.png} \\

Elizabeth M Wolkovich\\
Associate Professor of Forest \& Conservation Sciences\\ 
University of British Columbia
\end{letter}
\end{document}



% \signature{Elizabeth M Wolkovich}
\address{Forest and Conservation Sciences\\
University of British Columbia\\
2424 Main Mall\\
Vancouver, BC V6T 1Z4}


% including the complexity in measuring it and how it may structure communities in stationary and non-stationary systems. We've been working on a version of the storage effect model that gives us some interesting insights via simulations and I think a Review \& Synthesis where we marry these results with some of the long-term and experimental data available now could help advance the field.

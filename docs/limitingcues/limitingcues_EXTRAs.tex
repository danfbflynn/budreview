
Measuring the three major cues and understanding how they will interactively determine current and future phenology is hard for several major reasons. First, these cues vary across species, and possibly within species across the range  \citep{vitasse2009,harrington2015}. Second, the cues often interact: one cue may compensate for another cue \citep{Chuine2000}. In practice this means studies that are not designed to tease apart these cues experimentally (e.g., observational studies) may find statistical evidence for only one cue, because it masks one or more other cues (ADDCITES). For example, a plant that has not received enough chilling would generally require more forcing and thus leafout later, which could be indistinguishable (in observational data) from a photoperiod requirement that has not been met. \\

To some extent these interactions between cues have meant that---for many locations in most years---researchers have not been challenged to measure cues beyond forcing to explain advances and make accurate near-term predictions. In many locations if chilling cues are met and forcing produces budburst at an acceptably long photoperiod---then the main cue needed for predictions is forcing alone. In such cases, simple linear models of spring warming versus spring phenology may suffice \citep[e.g.,]{Ellwood2012}. But, increasing evidence, in some systems in some years, suggests that these chilling and photoperiod may play an important role in spring phenology today, with that importance growing in the future \citep{chuine2016,gauzere2019}. If rising evidence suggests these cues are critical to understand current responses and predict future non-linearities, then how do we integrate them more into phenological research and forecasting? \\% Thus, the cues in observational data---even long-term records---can be very hard to measure. 
% Dan thought the text worked through here...


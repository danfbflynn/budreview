\documentclass[11pt,letter]{article}
\usepackage[top=1.00in, bottom=1.0in, left=1.1in, right=1.1in]{geometry}
\renewcommand{\baselinestretch}{1.1}
\usepackage{graphicx}
\usepackage{natbib}
\usepackage{gensymb}
\usepackage{amsmath}
\usepackage{lineno}


\def\labelitemi{--}
\parindent=0pt

\usepackage{Sweave}
\begin{document}
\bibliographystyle{/Users/Lizzie/Documents/EndnoteRelated/Bibtex/styles/besjournals}
\renewcommand{\refname}{\CHead{}}
\begin{flushright}
Version dated: \today
\end{flushright}
\bigskip
\noindent RH: Interactive cues and spring phenology
% put in your own RH (running head)

\bigskip
\medskip
\begin{center}

% Insert your title:
\noindent{\Large {\bf Limiting cues: How spring warming, winter chilling and daylength will shape climate change responses}}\\ % OR How interactive cues will shape climate change responses ? 
% Concept paper on understanding interactive cues and climate change (with growth chamber studies) 
\bigskip

\noindent {\normalsize \sc
The lab in 2017$^{1-5}$}\\ % Lizzie, Ailene, Cat, Dan, Nacho
\noindent {\small \it
$^1$ Forest \& Conservation Sciences, Faculty of Forestry, University of British Columbia, 2424 Main Mall, Vancouver, BC V6T 1Z4\\
$^2$ Arnold Arboretum of Harvard University, 1300 Centre Street, Boston, Massachusetts, 02131, USA\\
$^3$ Organismic \& Evolutionary Biology, Harvard University, 26 Oxford Street, Cambridge, Massachusetts, 02138, USA\\
$^4$ The Nature Conservancy, 74 Wall Street, Seattle, Washington USA\\
$^5$ Global Change Ecology and Evolution Group, Department of Life Sciences, University of Alcal\`a, Alcal\`a de Henares 28805, Spain}
\end{center}
\medskip
\noindent{\bf Corresponding author:} Lizzie, see $^{3}$ above ; E-mail: e.wolkovich@ubc.ca\\

% A. K. Ettinger
% C. J. Chamberlain
% D. M. Buonaiuto
% I. Morales-Castilla


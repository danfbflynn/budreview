\documentclass[11pt,letterpaper]{article}

\usepackage{fixltx2e}
\usepackage{textcomp}
\usepackage{fullpage}
\usepackage{amsfonts}
\usepackage{verbatim}
\usepackage[english]{babel}
\usepackage{pifont}
\usepackage{color}
\usepackage{setspace}
\usepackage{lscape}
\usepackage{indentfirst}
\usepackage[normalem]{ulem}
\usepackage{booktabs}
%\usepackage{nag}
\usepackage{natbib}
%\usepackage{bibtex}
\usepackage{float}
\usepackage{latexsym}
%\usepackage{hyperref} 
\usepackage{url}
%\usepackage{html}
\usepackage{hyperref}
\usepackage{epsfig}
\usepackage{graphicx}
\usepackage{amssymb}
\usepackage{amsmath}
\usepackage{bm}
\usepackage{array}
\usepackage{mhchem}
\usepackage{ifthen}
\usepackage{caption}
\usepackage{hyperref}
%\usepackage{xcolor}
\usepackage{amsthm}
\usepackage{amstext}
\usepackage{lineno}

\usepackage{sectsty,setspace,natbib}
\usepackage[top=1.00in, bottom=1.0in, left=1in, right=1.25in]{geometry}
\usepackage{graphicx}
\usepackage{latexsym,amssymb,epsf,rotating}
\usepackage{epstopdf}
\usepackage{amsmath}
\usepackage{natbib}
\usepackage{todonotes}
\usepackage{framed}

\linespread{1.2} % was 1.66 for double-spaced 
% \raggedright
\setlength{\parindent}{0.5in}

\setcounter{secnumdepth}{0}
% Our sections are not numbered and our papers do not have
% Tables of Contents. We don't 
% present a list of figures or list of tables, either.

% Any common font is fine.
% (A common sans-serif font should be used on figures, but figures should be
% separate from the LaTeX document.)

\pagestyle{empty}

\renewcommand{\section}[1]{%
\bigskip
\begin{center}
\begin{Large}
\normalfont\scshape #1
\medskip
\end{Large}
\end{center}}

\renewcommand{\subsection}[1]{%
\bigskip
\begin{center}
\begin{large}
\normalfont\itshape #1
\end{large}
\end{center}}

\renewcommand{\subsubsection}[1]{%
\vspace{2ex}
\noindent
\textit{#1.}---}

\renewcommand{\tableofcontents}{}

\bibpunct{(}{)}{;}{a}{}{,}  % this is a citation format command for natbib

\parskip=5pt
\pagenumbering{arabic}
\pagestyle{plain}

\begin{document}
\begin{flushright}
Version dated: \today
\end{flushright}
\bigskip
\noindent RH: Interactive cues and spring phenology
% put in your own RH (running head)

\bigskip
\medskip
\begin{center}

% Insert your title:
\noindent{\Large {\bf Concept paper on understanding interactive cues and climate change (with growth chamber studies)}}\\
\vspace{2ex}
{\Large How interactive cues will drive climate change responses \\
\vspace{2ex}
Spring warming, winter warming or daylength: What cue will be most limiting in future tree phenology?} % What cue will be most limiting with climate change? Forcing x photoperiod x chilling
\bigskip

\noindent {\normalsize \sc
The lab$^{1,2}$}\\
\noindent {\small \it
$^1$ Arnold Arboretum of Harvard University, 1300 Centre Street, Boston, Massachusetts, 02131, USA\\
$^2$ Organismic \& Evolutionary Biology, Harvard University, 26 Oxford Street, Cambridge, Massachusetts, 02138, USA\\
$^3$ Forest \& Conservation Sciences, Faculty of Forestry, University of British Columbia, 2424 Main Mall, Vancouver, BC V6T 1Z4}\\

\end{center}
\medskip
\noindent{\bf Corresponding author:} XX, see $^{1,2}$ above ; E-mail:.\\

\newpage
%\linenumbers

\begin{abstract}
Climate change has shifted plant phenology globally, with average shifts of 4-6 days/\textdegree C and some species shifting several weeks. Globally, such shifts have been some of the most reported and most predictable biological impacts of climate change. This predictability comes from decades of research, which have outlined the major cues that drive most studied plant phenology: temperatures (including spring warming and winter chilling) and daylength. Further simplifying predictions, spring temperatures are often the dominant cue in nature, making linear models of heat sums often excellent at predicting interannual variation in phenology. Yet as climate change has marched on, new research has uncovered failures to predict the current observed changes, with many shifts appearing more muted over certain time periods or in certain locations. Here we argue that such inaccurate predictions are most likely due to simple models that neglect to consider other major cues---especially winter chilling and daylength, which moderate and shape plant phenological responses to spring warming. We highlight how over 60 years of research in controlled environments can improve predictions for when, where and how the interactive effects of other cues will impact simple linear predictions. Finally, we discuss how a new generation of controlled environment experiments could rapidly improve our predictive capacity for woody plant phenology in coming decades.  
\end{abstract}

\noindent \emph{Main message (and, really, it's important):} if you want to project climate change impacts, you need to focus on relevant changes in all three cues. The relevant changes part is about comparing cues, the all three cues is about interactive cues.

\noindent \emph{Keywords:} phenology, climate change, spring warming, forcing, chilling, daylength, photoperiod, non-linear responses, leafout, budburst\\

\newpage
\noindent {\bf Questions}, which are in addition to notes in orange throughout and general thoughts/ideas for improvement! Thanks for all your help!
\begin{enumerate}
\item Should we finalize that we won't do the PhenoFit figure? As a reminder, the plan was to calculate the current forcing, chilling, photoperiod and the future of each of these. But we don't have the climate data....
\item Can we suggest where cues other than forcing are most critical? Perhaps at least for our focal spp.? If so, where would we add this (1.c)?
\begin{enumerate}
\item If we know where a non-linearity is in a cue (e.g., from experiments of models), then wherever you're near that on a range, you should expect bigger effects of that cue.
\item Maybe where chilling will change to above versus below the threshold that plants can sense. 
\end{enumerate}
\item Is the intro too long? If so, please suggest ways to cut and/or re-arrange.
\end{enumerate}

\section{Outline}
% Ask yourself: What do people need to do to fit better models and predict limits to shifts?
\begin{enumerate}
\item Introduction 
\begin{enumerate}
\item Shifts in spring phenology are one of the most reported and most predictable changes with climate change
\begin{enumerate}
\item Review the shifts across space and time and how coherent they are
\item Outline some studies showing how predictable they are (Spring Index? Primack linear paper)\todo[inline]{What other citations/examples can we add?}
\end{enumerate}
\item Recently however predictions have started to fail in certain places or over certain time periods
\begin{enumerate}
\item Fu paper, Tibet etc., \todo[inline]{other citations?}
\item The main hypothesis for this failure is other cues most observational studies have ignored
\end{enumerate}
\item There has been a lot of focus on forcing but really it's more complicated
\begin{enumerate}
\item Think about two example tree species: \emph{Fagus sylvatica, Betula pendula}
\item Look at their distributions
\item Cues are adapted to high climate variation! \todo[inline]{New term: Cue range limits. What do you think? As in the limits of cues as seen over a species' range, what do you think?}
\item For many species three major cues drive spring phenology: forcing, chilling, daylength
\item Research in this has been especially strong for woody species phenology: this is where we understand things best and thus our main focus here, though much of what we say could apply to non-woody species with similar cues (cite \emph{Arabidopsis}).\todo{What other herbaceous species can we cite here? Should we touch on seed germination?}
\end{enumerate}
\item These cues may create critical non-linear responses that most current methods cannot predict.
\item However, measuring these cues and thinking about how they will interactively produce future phenology is hard because:
\begin{enumerate}
\item They are expected to interact; cues may compensate for other cues; meaning they mask one another (e.g., chilling cue not fully being met could look like a photoperiod requirement that has not been met)
\item They vary across species and possibly within species across the range
\item They are hard to measure.
\item To some extent, we haven't really had to measure these other cues to get decent predictions for lots of places and years
\end{enumerate}
\item But if evidence is rising that these cues are critical, how do we integrate them more into phenological research? Step 1 is clearly to figure out how to robustly measure them. 
\begin{enumerate}
\item Methods especially lame at understanding these cues (and thus predicting non-linearities): models from long-term observational data ... somehow mention experiments maybe?
\item Comment on the two issues at play here: the data type (e.g., long-term) versus the model type (e.g., linear and sans interactions?)
\item The one method designed to look at all these cues is controlled environment (generally growth chamber) studies
\end{enumerate}
\item Growth chamber studies
\begin{enumerate}
\item Can manipulate all three cues (and even more, humidity etc. nod?)
\item Are often focused on interactions (unlike other methods)
\item Have been done \emph{forever}. But oddly, never really reviewed.
\item  ...and are often poorly integrated into current climate change literature. Including debates where they are critical, like about photoperiod. 
\end{enumerate}
\item Our aim is to:
\begin{enumerate}
\item Review how three major phenological cues for woody plant phenology will shift in coming decades with anthropogenic climate change
\item Review of the three major phenological cues from growth chamber studies over the past 60 (70?) years
\item Highlight their critical relevance to climate change research \todo[inline]{Should we cut this? Already covered above?}
\item Compare treatments from controlled environment studies to predicted shifts in cues with climate change.  
\item Showcase how growth chamber studies can be best designed to better understand these interactive cues (paths forward). 
\end{enumerate}
\end{enumerate}
\item Review how cues will shift with climate change (here we show the figures that Nacho has produced for two PEP725 spp.)
\begin{enumerate}
\item Forcing: the world will get warmer %\todo[inline]{This got me wondering if we should ask Ben Cook if he would collaborate.}
\begin{enumerate}
\item Higher altitude and arctic places will warm more
\item Give range of warming depending on different scenarios
\item Minima warm more than maxima (night-time temps)
\item Different seasons may warm differently
\end{enumerate}
\item Chilling, see forcing but ... 
\begin{enumerate}
\item Chilling only occurs between certain temps so some places accumulate more chilling with warming
\item And there is so much we don't know about how chilling works and interacts with forcing (sequential model, parallel models etc.)
\end{enumerate}
\item Photoperiod: Shifts with phenology
\begin{enumerate}
\item Changes in forcing and chilling will alter the photoperiod that matters so to speak
\item Need a little more here ... \todo[inline]{Help appreciated on what to add for daylength}
\end{enumerate}
\end{enumerate}
\item Compare treatments from controlled environment studies to predicted shifts in cues with climate change: Part I: Review of the three major phenological cues from growth chamber studies over the last 67 years % 2014-1947
\begin{enumerate}
\item Be sure to note that most studies were \emph{not} done for climate change research, they were done for fundamental science or agricultural purposes \todo{Any other reasons for most woody species phenology growth chamber studies?}
\item Quick intro to the data, how long, which cues .... 
\begin{enumerate}
\item Fig: Number of studies by year (OSPREE)
\item Fig: Map of studies, color coded or such by which of the three cues they manipulated
\end{enumerate}
\item For each of the three cues:
\begin{enumerate}
\item  X\% of studies manipulated that cue
\item Variation across space, continent and time (and species)? 
\item Fig: Variation in treatments across space (photo/chill/force)
\item Fig: Variation in treatments across time (graph with year on $x$-axis or divide time in half or such? 
\item Say something about material (seeds/saplings/cuttings)? Can we tie to relevance of predicting future forest communities or such?
\end{enumerate}
\item X\% of studies manipulated which interacting cues? (i.e., how many studies manipulate 1 cues, 2 cues, 3 cues ... of those manipulating 1 cue, what is the breakdown by cue etc.)
\end{enumerate}
\item Compare treatments from controlled environment studies to predicted shifts in cues with climate change: Part II: What cues will be most limiting with climate change? How do controlled environment studies compare? 
\begin{enumerate}
\item Consider both the range of a species and the climate change projections ...
\begin{enumerate}
\item Take each PEP725 datapoint within our selected species' range and calculate:
\begin{enumerate}
\item Min daily temp for 1-2 months before leafout (Nacho's figures)
\item Max daily temp for 1-2 months before leafout (Nacho's figures) \footnote{We used daily min/max, as they're most directly comparable to OSPREE.}
\end{enumerate}
\item Add potential PhenoFit figure here? 3D of all three cues and how they shifted?
\end{enumerate}
\end{enumerate}
\item Paths forward (showcase how growth chamber studies can be best designed to better understand these interactive cues in regards to climate change forecasting) \todo[inline]{I think this section needs the most help! We have a lot of points and I think we need to clarify them and cut some probably.}
\begin{enumerate}
\item Consider the following when designing experiments:
\begin{enumerate}
\item Consider the cues with the current vs. future range of a species (as we did above) to inform experiments
\item If you don't work within the range or projected cue range limits of a species, then consider working on informative extremes...
\item So consider thermal tolerances/limits (and where is the species optimum?) ...These could be especially useful for understanding range limits: look at treatments beyond the variation seen within a species' range and see if there is abrupt change or you see continuous change, if no abrupt change then it may suggest that something else must limit range (e.g., biotic cues, minima temperature after which species)
\item After you carefully select the cues to study, make your reasoning clear \todo[inline]{do we need this?}
\end{enumerate}
\item Manipulating one cue may be less useful, esp. if we want to advance comparisons with long-term data
\item Major need to better understand deviations in long-term data are: better non-linear models for more species. How best to do this?
\begin{enumerate}
\item Studies using only long-term observational data must test for and be up front about correlations in predictor variables (e.g., chilling, forcing and daylength often covary) and thus what they can and cannot rule out. \todo[inline]{do we need this?}
\item A better option than just long-term data are more efforts to integrate long-term data with growth chamber studies.
\begin{enumerate}
\item Studies that test the extremes are needed to parameterize models (ideally you need to know where the zeros are).
\item Traditional methods to hold-out data and test how well the model performs
\item Use growth chamber studies to test model predictions, especially in future climate scenarios where non-linearities are predicted. 
\item Improving models means more back-and-forth worth between developing models based on both long-term data and experiments, then testing predictions with new experiments and as newer observational data are generated (i.e., more years and also data from new locations)
\end{enumerate}
\end{enumerate}
\item Studies not interested in climate change forecasting can still contribute---with little effort---to progress in this area by: Reporting all cues (even the ones you don't measure) so they can be used in modeling efforts. 
\end{enumerate}
\item Wrap-up.... Climate change: it means all that work on phenology comes due ... now!
\item Now that climate change is here, experiments that want to claim climate change relevance must address what the cue impacts of climate change will really be
\end{enumerate}



\newpage
\section{Figures}
\begin{enumerate}
\item PEP725 spp. 1980 figures
\item PEP725 spp. Future figures
\item Number of studies by year (OSPREE) \emph{Other ideas?!} Such as, number of species studied by year. Show crops or remove or show separately?
\item Map of studies, color coded or such by which of the three cues they manipulated
\item Variation in treatments across space (photo/chill/force)
\item Variation in treatments across time (graph with year on $x$-axis or divide time in half or such? 
\item Not a figure, but analysis-related: X\% of studies manipulated which interacting cues? (i.e., how many studies manipulate 1 cues, 2 cues, 3 cues ... of those manipulating 1 cue, what is the breakdown by cue etc.)
\end{enumerate}




%=======================================================================
% \section{}
%=======================================================================

%=======================================================================
%\section{Acknowledgements}
%=======================================================================



%=======================================================================
% References
%=======================================================================
%\newpage
%\bibliography{PWRminimal}
%\bibliographystyle{sysbio.bst}


%=======================================================================
% Tables
%=======================================================================

%\begin{center}  
%\begin{table}
%\caption{Key differences between PWR and traditional PCMs such as PGLS.}
%\begin{tabular}{ | p{4cm} | p{5.5 cm} | p{5.5 cm} |}   \hline 
%& PWR & PCMs (e.g., PGLS) \\ \hline \hline
%Major goal & Study of evolution of correlation between variables across species & Study of evolution of correlation between variables across species\\ \hline
%\emph{Assumption 1:} Nature of correlation between two or more variables & Non-stationary (changes through phylogeny in a phylogenetically conserved fashion) & Stationary (constant) throughout phylogeny (all variation is noise) \\ \hline
%\emph{Assumption 2:} Completeness of variables & Substitutes phylogeny for variables (simple or complex) not in the model that interact with variables in the model & Assumes variables in model are primary drivers of correlational relationship \\ \hline
%Inferential mode & Usually exploratory & Hypothesis testing (statistical significance)\\ \hline
%Outputs & Coefficients of regression changing through the phylogeny & p-value and single set of coefficients presumed to apply to entire phylogeny with their confidence intervals\\ \hline

%Method to avoid overfitting & Cross-validation (boot-strapped determination of optimal band-width for accurate prediciton of hold-outs) & Exact analytical model of errors and degrees of freedom\\ \hline \hline
%\end{tabular}
%\end{table}
%\end{center}

\newpage

%=======================================================================
% Figures
%=======================================================================




\end{document}
%%%%%%%%%%%%%%%%%%%%%%%%%%%%%%%%%%%%%%%%%%%%%%%%%%%%%%%%%%%%%%%%%%%%%%%%

% Example figure

\begin{figure}[t!]
\centering
\includegraphics[width=1\textwidth]{arnell-pwr-m.pdf}
\caption{PWR estimates (red dots) and 95\% confidence intervals (red lines) across the Arnell phylogeny using O-U-based distance  weighting, for a simple regression of first flowering date on seed size. Solid and dashed vertical gray lines show the estimate and 95\% CI from an analogous global model estimated using PGLS.}
  \label{fig:arnell-pwr}
\end{figure}
\clearpage


%=======================================================================
% to-do listing
%=======================================================================

\listoftodos

%=======================================================================
\section*{Other loose ends}
%=======================================================================
\documentclass[11pt,letterpaper]{article}

\usepackage{fixltx2e}
\usepackage{textcomp}
\usepackage{fullpage}
\usepackage{amsfonts}
\usepackage{verbatim}
\usepackage[english]{babel}
\usepackage{pifont}
\usepackage{color}
\usepackage{setspace}
\usepackage{lscape}
\usepackage{indentfirst}
\usepackage[normalem]{ulem}
\usepackage{booktabs}
%\usepackage{nag}
\usepackage{natbib}
%\usepackage{bibtex}
\usepackage{float}
\usepackage{latexsym}
%\usepackage{hyperref} 
\usepackage{url}
%\usepackage{html}
\usepackage{hyperref}
\usepackage{epsfig}
\usepackage{graphicx}
\usepackage{amssymb}
\usepackage{amsmath}
\usepackage{bm}
\usepackage{array}
\usepackage{mhchem}
\usepackage{ifthen}
\usepackage{caption}
\usepackage{hyperref}
%\usepackage{xcolor}
\usepackage{amsthm}
\usepackage{amstext}
\usepackage{lineno}

\usepackage{sectsty,setspace,natbib}
\usepackage[top=1.00in, bottom=1.0in, left=1in, right=1.25in]{geometry}
\usepackage{graphicx}
\usepackage{latexsym,amssymb,epsf,rotating}
\usepackage{epstopdf}
\usepackage{amsmath}
\usepackage{natbib}
\usepackage{todonotes}
\usepackage{framed}

\linespread{1.2} % was 1.66 for double-spaced 
% \raggedright
\setlength{\parindent}{0.5in}

\setcounter{secnumdepth}{0}
% Our sections are not numbered and our papers do not have
% Tables of Contents. We don't 
% present a list of figures or list of tables, either.

% Any common font is fine.
% (A common sans-serif font should be used on figures, but figures should be
% separate from the LaTeX document.)

\pagestyle{empty}

\renewcommand{\section}[1]{%
\bigskip
\begin{center}
\begin{Large}
\normalfont\scshape #1
\medskip
\end{Large}
\end{center}}

\renewcommand{\subsection}[1]{%
\bigskip
\begin{center}
\begin{large}
\normalfont\itshape #1
\end{large}
\end{center}}

\renewcommand{\subsubsection}[1]{%
\vspace{2ex}
\noindent
\textit{#1.}---}

\renewcommand{\tableofcontents}{}

\bibpunct{(}{)}{;}{a}{}{,}  % this is a citation format command for natbib

\parskip=5pt
\pagenumbering{arabic}
\pagestyle{plain}

\begin{document}
\begin{flushright}
Version dated: \today
\end{flushright}
\bigskip
\noindent RH: Interactive cues and spring phenology
% put in your own RH (running head)

\bigskip
\medskip
\begin{center}

% Insert your title:
\noindent{\Large \bf Concept paper on understanding interactive cues and climate change (with growth chamber studies); or How interactive cues will drive climate change responses} % What cue will be most limiting with climate change? Forcing x photoperiod x chilling
\bigskip

\noindent {\normalsize \sc
The lab$^{1,2}$}\\
\noindent {\small \it
$^1$ Arnold Arboretum, 1300 Centre Street, Boston, Massachusetts, 02131, USA\\
$^2$ Organismic \& Evolutionary Biology, 28 Oxford Street, Cambridge, Massachusetts, 02138, USA}\\
\end{center}
\medskip
\noindent{\bf Corresponding author:} XX, see $^{1,2}$ above ; E-mail:.\\

\newpage
%\linenumbers

\subsubsection{Abstract} Goes here.
\\
\noindent (Keywords: phenology, climate change, spring warming, forcing, chilling, daylength, photoperiod, non-linear responses, leafout, budburst)\\

\newpage

% Ask yourself: What do people need to do to fit better models and predict limits to shifts?
\begin{enumerate}
\item Introduction 
\begin{enumerate}
\item Climate change: it means all that work on phenology comes due ... now! (Short opening paragraph)
\item There has been a lot of focus on forcing but really it's more complicated
\begin{enumerate}
\item Think about one tree\footnote{Here we pick one PEP725 species that is well-represented spatially for leafout or budburst data (Cat?)}
\item Look at its distribution!\footnote{Here, we show a distribution map (Nacho?), maybe with some spring climate and/or phenology mapped on it.}
\item Cues are adapted to high climate variation!
\end{enumerate}
\item These cues may create critical non-linear responses that most current methods cannot predict, but measuring them and thinking about how they will interactively produce future phenology is hard because:
\begin{enumerate}
\item They are expected to interact; cues may compensate for other cues; meaning they mask one another
\item They vary across species and possibly within species across the range
\item They are hard to measure.\footnote{Somewhere in here need to sneak in that we will focus on woody species phenology, because it's where we understand things best and thus should build from there.}
\end{enumerate}
\item How do you measure them there cues?
\begin{enumerate}
\item Methods especially lame at understanding these cues (and thus predicting non-linearities): models from long-term observational data ... somehow mention experiments maybe?
\item Try to comment on the two issues at play here: the data type (e.g., long-term) versus the model type (e.g., linear and sans interactions?)
\item The one method designed to look at all these cues is controlled environment (generally growth chamber) studies
\end{enumerate}
\item Growth chamber studies
\begin{enumerate}
\item Can manipulate all three cues (and even more, humidity etc. nod?)
\item Are often focused on interactions (unlike other methods)
\item Have been done \emph{forever}. But oddly, never really reviewed.
\item  ...and are often poorly integrated into current climate change literature. Including debates where they are critical, like about photoperiod.
\end{enumerate}
\item Our aim is to:
\begin{enumerate}
\item Review how three major phenological cues for woody plant phenology will shift in coming decades with anthropogenic climate change
\item Review of the three major phenological cues from growth chamber studies over the past 60 (70?) years
\item Highlight their critical relevance to climate change research
\item Compare treatments from controlled environment studies to predicted shifts in cues with climate change.  
\item Showcase how growth chamber studies can be best designed to better understand these interactive cues. 
\end{enumerate}
\end{enumerate}
\item Review how cues will shift with climate change (we could have figures of change in temp across the distibution here?)
\begin{enumerate}
\item Forcing: the world will get warmer %\todo[inline]{This got me wondering if we should ask Ben Cook if he would collaborate.}
\begin{enumerate}
\item Higher altitude and arctic places will warm more
\item Give range of warming depending on different scenarios
\item Minima warm more than maxima (night-time temps)
\end{enumerate}
\item Chilling, see forcing but ... 
\begin{enumerate}
\item Chilling only occurs between certain temps so some places accumulate more chilling with warming
\item And there is so much we don't know about how chilling works and interacts with forcing (sequential model, parallel models etc.)
\end{enumerate}
\item Photoperiod: Shifts with phenology
\begin{enumerate}
\item Changes in forcing and chilling will alter the photoperiod that matters so to speak
\item Need more here ... 
\end{enumerate}
\end{enumerate}
\item Review of the three major phenological cues from growth chamber studies over the last 67 years % 2014-1947
\begin{enumerate}
\item Quick intro to the data, how long, which cues .... 
\begin{enumerate}
\item Fig: Number of studies by year (OSPREE)\footnote{Other ideas: number of species studied by year. Show crops or remove or show separately?}
\item Fig: Map of studies, color coded or such by which of the three cues they manipulated
\end{enumerate}
\item For each of the three cues:
\begin{enumerate}
\item  X\% of studies manipulated that cue
\item Variation across space, continent and time (and species)? 
\item Fig: Variation in treatments across space (photo/chill/force)
\item Fig: Variation in treatments across time (graph with year on $x$-axis or divide time in half or such? 
\end{enumerate}
\item X\% of studies manipulated which interacting cues? (i.e., how many studies manipulate 1 cues, 2 cues, 3 cues ... of those manipulating 1 cue, what is the breakdown by cue etc.)
\end{enumerate}
\item Random other bits to fit in
\begin{enumerate}
\item Say something about material (seeds/saplings/cuttings)?
\item We need better non-linear models.
\end{enumerate}
\item What cues will be most limiting with climate change?
\begin{enumerate}
\item Take each PEP725 datapoint within our selected species' range ... 
\item Calculate: \todo[inline]{We need to think about this a lot more! Do we want to do? And if so, exactly what metrics do we want?}
\begin{enumerate}
\item Min daily temp for 1-2 months before leafout
\item Max daily temp for 1-2 months before leafout
\item Chilling units (which?) for Oct-February ... actually, if we just want to do simple comparisons to the OSPREE data (and I think we do), then we could just do min/max temperatures (and mean?) for Oct-February I think!
\item Most directly comparable to OSPREE would be daily min and max temperatures I think; might also be important to use daily min/max if we are focused so closely on getting things accurate. But we should discuss how feasible this is. 
\end{enumerate}
\end{enumerate}
\item Wrap-up....
\end{enumerate}


%=======================================================================
% \section{}
%=======================================================================

%=======================================================================
%\section{Acknowledgements}
%=======================================================================



%=======================================================================
% References
%=======================================================================
%\newpage
%\bibliography{PWRminimal}
%\bibliographystyle{sysbio.bst}


%=======================================================================
% Tables
%=======================================================================

%\begin{center}  
%\begin{table}
%\caption{Key differences between PWR and traditional PCMs such as PGLS.}
%\begin{tabular}{ | p{4cm} | p{5.5 cm} | p{5.5 cm} |}   \hline 
%& PWR & PCMs (e.g., PGLS) \\ \hline \hline
%Major goal & Study of evolution of correlation between variables across species & Study of evolution of correlation between variables across species\\ \hline
%\emph{Assumption 1:} Nature of correlation between two or more variables & Non-stationary (changes through phylogeny in a phylogenetically conserved fashion) & Stationary (constant) throughout phylogeny (all variation is noise) \\ \hline
%\emph{Assumption 2:} Completeness of variables & Substitutes phylogeny for variables (simple or complex) not in the model that interact with variables in the model & Assumes variables in model are primary drivers of correlational relationship \\ \hline
%Inferential mode & Usually exploratory & Hypothesis testing (statistical significance)\\ \hline
%Outputs & Coefficients of regression changing through the phylogeny & p-value and single set of coefficients presumed to apply to entire phylogeny with their confidence intervals\\ \hline

%Method to avoid overfitting & Cross-validation (boot-strapped determination of optimal band-width for accurate prediciton of hold-outs) & Exact analytical model of errors and degrees of freedom\\ \hline \hline
%\end{tabular}
%\end{table}
%\end{center}

\newpage

%=======================================================================
% Figures
%=======================================================================




\end{document}
%%%%%%%%%%%%%%%%%%%%%%%%%%%%%%%%%%%%%%%%%%%%%%%%%%%%%%%%%%%%%%%%%%%%%%%%

% Example figure

\begin{figure}[t!]
\centering
\includegraphics[width=1\textwidth]{arnell-pwr-m.pdf}
\caption{PWR estimates (red dots) and 95\% confidence intervals (red lines) across the Arnell phylogeny using O-U-based distance  weighting, for a simple regression of first flowering date on seed size. Solid and dashed vertical gray lines show the estimate and 95\% CI from an analogous global model estimated using PGLS.}
  \label{fig:arnell-pwr}
\end{figure}
\clearpage


%=======================================================================
% to-do listing
%=======================================================================

\listoftodos

%=======================================================================
\section*{Other loose ends}
%=======================================================================
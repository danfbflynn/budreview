\documentclass{article}\usepackage[]{graphicx}\usepackage[]{color}
% maxwidth is the original width if it is less than linewidth
% otherwise use linewidth (to make sure the graphics do not exceed the margin)
\makeatletter
\def\maxwidth{ %
  \ifdim\Gin@nat@width>\linewidth
    \linewidth
  \else
    \Gin@nat@width
  \fi
}
\makeatother

\definecolor{fgcolor}{rgb}{0.345, 0.345, 0.345}
\newcommand{\hlnum}[1]{\textcolor[rgb]{0.686,0.059,0.569}{#1}}%
\newcommand{\hlstr}[1]{\textcolor[rgb]{0.192,0.494,0.8}{#1}}%
\newcommand{\hlcom}[1]{\textcolor[rgb]{0.678,0.584,0.686}{\textit{#1}}}%
\newcommand{\hlopt}[1]{\textcolor[rgb]{0,0,0}{#1}}%
\newcommand{\hlstd}[1]{\textcolor[rgb]{0.345,0.345,0.345}{#1}}%
\newcommand{\hlkwa}[1]{\textcolor[rgb]{0.161,0.373,0.58}{\textbf{#1}}}%
\newcommand{\hlkwb}[1]{\textcolor[rgb]{0.69,0.353,0.396}{#1}}%
\newcommand{\hlkwc}[1]{\textcolor[rgb]{0.333,0.667,0.333}{#1}}%
\newcommand{\hlkwd}[1]{\textcolor[rgb]{0.737,0.353,0.396}{\textbf{#1}}}%
\let\hlipl\hlkwb

\usepackage{framed}
\makeatletter
\newenvironment{kframe}{%
 \def\at@end@of@kframe{}%
 \ifinner\ifhmode%
  \def\at@end@of@kframe{\end{minipage}}%
  \begin{minipage}{\columnwidth}%
 \fi\fi%
 \def\FrameCommand##1{\hskip\@totalleftmargin \hskip-\fboxsep
 \colorbox{shadecolor}{##1}\hskip-\fboxsep
     % There is no \\@totalrightmargin, so:
     \hskip-\linewidth \hskip-\@totalleftmargin \hskip\columnwidth}%
 \MakeFramed {\advance\hsize-\width
   \@totalleftmargin\z@ \linewidth\hsize
   \@setminipage}}%
 {\par\unskip\endMakeFramed%
 \at@end@of@kframe}
\makeatother

\definecolor{shadecolor}{rgb}{.97, .97, .97}
\definecolor{messagecolor}{rgb}{0, 0, 0}
\definecolor{warningcolor}{rgb}{1, 0, 1}
\definecolor{errorcolor}{rgb}{1, 0, 0}
\newenvironment{knitrout}{}{} % an empty environment to be redefined in TeX

\usepackage{alltt}[12pt]
\usepackage{Sweave}
\usepackage{mdframed}
\usepackage{hyperref}
\topmargin -1.5cm        
\oddsidemargin -0.04cm   
\evensidemargin -0.04cm
\textwidth 16.59cm
\textheight 21.94cm 
\renewcommand{\baselinestretch}{2}
\parindent 0pt
\usepackage{setspace}
\singlespacing

\newmdenv[
  topline=true,
  bottomline=true,
  skipabove=\topsep,
  skipbelow=\topsep
]{siderules}
\IfFileExists{upquote.sty}{\usepackage{upquote}}{}
\begin{document}
\section*{How to scape new papers for OSPREE: July 2019}
\subsection*{Getting Started}
You'll need:
\begin{enumerate}
\item Excel or other program that makes .xls  or .csv files
\item ImageJ download for free from here \url{https://imagej.net/Welcome}. You'll also need to add the Figure\textunderscore Calibration.class, which will help for giving x and y calibrations to images. 
\begin{enumerate}
\item To add the the Figure\textunderscore Calibration.class: In ImageJ go to plugins, select \textit{add plug in}, then navigate to the Figure\textunderscore Calibration.class file that you downloaded on your computer, click on it and follow through a few clicks to add the plugin.
\item If you have some trouble getting the measurements to show up after calibrating, try switching to the pointer and clicking. You might need to set the preferences on your pointer tool to auto-measure.
\begin{enumerate}
\item A clarification on above from Tim Savas (original OPSREE lead data enterer): \textit{After doing the figure calibration and selecting the yellow pointer tool, you start clicking inside the figure but no points appear. The reason for this is that the "rectangle" you drew for the figure calibration is still masking the figure, and until you click out of it, you can't draw points under it. It's hard to see! So to get rid of the invisible rectangle, just click the mouse once outside of its edge. Side note: After drawing all of your points onto a figure, you can press Command-M to bring up the resulting table of values. I do this in the video tutorial, and whenever I scrape, but didn't describe the key command!}
\end{enumerate}
\end{enumerate}
\end{enumerate}

Now here's what to do:
\begin{enumerate}
\item Copy the excel file ospree\textunderscore newpapers for git repo (ospree/data/ospree\textunderscore 2019update) and make your own extension, for example, Dan would write ``ospree\textunderscore 2019update\textunderscore dmb". This will be the spread sheet you enter your data into and then in the future, someone will merge all of our files together into the master data.
\item Familiarize yourself with each tab:
  \begin{enumerate}
  \item \textbf{meta\_general}: metadata for each sheet
  \item \textbf{source}: list of the paper we are working with. Bibliographic information and notes on usefulness for our purposes. Note the "ToDo" column, which tells you which figure or table to focus on. You may find other figures are better, these were from our initial quick read. Also pay attention to datasetID column, which tells you how you should enter the identifying information for each paper.
  \item \textbf{study}: Details on each experiment within each paper; possibly only one line for a paper, if only one experiment is relevant. This sheet is useful for our overview of what kind of experimental manipulations were done.
  \item \textbf{data\_detailed}: Detailed data for the experiment, with all relevant information filled out.
  \item Responses may be pre/post treatment, time, or other. Temporal responses, such as days to 50\% budburst, are fairly common. An example of an other type of response would be percent budburst, again fairly common.
  \item \textbf{scratch}: For temporary formatting and manipulating data scraped from ImageJ.
  \item The two most important tabs to fill out are \textbf{study} and \textbf{data\_detailed}. 
  \end{enumerate}
\item Read your paper and fill out the information in the ``study" and ``data\textunderscore detailed tab." Be sure your datasetID and study info agrees with the source tab, if not---figure out what is wrong and fix it.
\item Find the designated figure or table as noted in the source tab of the worksheet. Also note the datasetID column in source; use this as you fill out data\_detailed and study.
\item Take a screen shot of the figure and import into ImageJ, following Tim's instructions from lab meeting with a video of screen shots from git/ospree/notes/howtoscrape/Data Scraping Tutorial.mp4. Use the scratch tab to get data into the right format, and then copy into data\textunderscore detailed. Fill out the study tab as best as possible to describe the experimental treatments used in each study within each publication.
\end{enumerate}
\subsection*{A few more ``how to's", trouble shooting etc} 
 \begin{itemize}
 \item \textbf{For dealing with lats and lons:} Tool to convert to decimal latitude and longitude:\url{https://andrew.hedges.name/experiments/convert_lat_long/}  and remember to add NEGATIVE to your longitude if it's West. Also, you can check where things are by just typing in lat and long into Google maps.
\item \textbf{ For calculating the number of days between too dates:} Use this:\url{http://www.timeanddate.com/date/duration.html} to calculate the number of days between two dates for calculating dormancy. Note do NOT include the end date in the calculations.
\item \textbf{On entering response times:} If the respvar of a study is "daystobudburst" enter a 1 into the "response" column, and fill in the recorded time in the "response.time" column.
\item \textbf{On dealing with error:} If it's figures records error and it is *clear enough* to scrape, error can be recorded. Often times the SE bars are in the way of each other or not quite discernible, in which case we've decided to avoid them. But if the bars are clear, record them. Values can go in "resp\textundersccore error" and just SE in "error type."
\end{itemize}

\end{document}
